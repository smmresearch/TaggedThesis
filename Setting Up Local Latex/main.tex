\documentclass{article}
\usepackage{hyperref}
\begin{document}
Still a work in progress...

There are many ways to set-up LaTeX on your local machine.
My favorite is to use VS Code with Latex Workshop.
There are several guides on the internet for the basic set-up of VS Code with Latex Workshop.
But here are a couple things I have learned and found useful.

\subsection{Using Version Control}
Using version control is a great way to back-up your work and be able to see the history of the document.
VS Code has built-in support for GitHub, which you can find by choosing the version control tab on the left.
You will need a GitHub account (free; GitHub is owned for Microsoft).

For version control to be at its best, however, you don't any one line to be too long.
One way to accomplish this is to have every sentence on its own line (as LaTeX ignores single new-lines).
You can do this manually, or you can configure VS Code to automatically run a script called ``latexindent.pl'' (https://github.com/cmhughes/latexindent.pl) when you save the file to format your file for you. Occasionally, this adds line breaks where they should not be, but it a minor problem.

If you want to use the

\end{document}
