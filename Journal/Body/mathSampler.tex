\newcommand{\RipsD}{\operatorname{Rips}_1}
\newcommand{\Rips}{\operatorname{Rips}}
\newcommand{\ver}{\operatorname{ver}}
\newcommand{\diam}{\operatorname{Diam}}
\newcommand{\midR}{\operatorname{mid}}
\newcommand{\dN}{N_1^*}

\chapter{CHAPTER WITH MATH}
\begin{center}
  Authors and Affiliations \\
  Modified from a manuscript to be submitted to/ under review/ published in \textit{Name of the Journal}
  \blfootnote{A version of this chapter appears in Journal of Discipline, Volume 18, Issue 3}
\end{center}

\section{Abstract}
This is the text of my abstract that is part of the thesis itself.
The abstract describes the work in the first paper general. You can use the same abstract as your paper here.
\section{Proofs and Stuff}
\begin{definition}
  A set $A$ is something.
\end{definition}

\begin{lemma}
  If cool, then great.
  \label{lem2}
\end{lemma}
\begin{proof}
  Without loss of generality, it works.
  \begin{equation}
    \label{orthogonalProjectionIsGoodActually}
    d(x,y)= d(x,z)+d(z,y) \geq d(x,x-\langle x,n\rangle n )+0 = \langle x,n\rangle.
  \end{equation}
  Furthermore,
  \begin{align}
    \ell_1(\hat{x},y) & = \ell_1 (x,y)                                        \\
                      & =|\ell_1(x,y)-2\langle x,n\rangle \ell_1(n,0)|        \\
                      & \left(\frac{x+y+z}{2x+y}\right) - \left(2x^2-y\right) \\
                      & B\left\langle \frac{4}{x}+x^3\right\rangle            \\
                      & \Bigg(x^2-2x\Bigg)
  \end{align}

  From \autoref{orthogonalProjectionIsGoodActually}, it follows \[\ell_1(\hat{x},y)\ \ell_1(\hat{x},y)-2\langle x,n\rangle \leq\]
\end{proof}
Wasn't that a great proof of \autoref{lem2}?
\begin{lemma}
  A lemma.
\end{lemma}
\begin{proof}
  \[3x+4=12\]
\end{proof}

Then, we should also have some in-line math $B\left(\frac{3x}{2y-x}\right)$ and then $d(x,y)$ if it is alright. We might also have $\sqrt{x^2+\frac{3}{x}}$.

\begin{theorem}
  If true, then it all collapses.
  \label{WhatAGreatTheorem}
\end{theorem}
\begin{proof}
  By Zorn's lemma, Zorn has the best name \autocite{martiniCompleteReducedConvex2019}.
  Also, \autocite{chenGraphHomotopyGraham2001} and \autocite{dochtermannMinimalGraphsContractible2023}.
  \[x^2+y^2+x^2=2.\]

\end{proof}
\section{Floating Practice}
Text here.

%similar alt text as tikz works with e.g. \begin{tikzcd} or \begin{picture}

%a tikz picture that will not move
\begin{center}
  \captionsetup{type=figure}%if you don't using floating figures, you must tell latex that it is a figure
  \begin{tikzpicture}[alt={tikz graphic that works natively on the June version}]
    \draw[gray, thick] (-1,2) -- (2,-4);
    \draw[gray, thick] (-1,-1) -- (2,2);
    \filldraw[black] (0,0) circle (2pt) node[anchor=west]{A Tikz Picture that stays put! };
  \end{tikzpicture}
  \caption{A caption for a tikz picture}
\end{center}
Some more text here you see.
%a floating tikz instead
\begin{figure}[tb]
  \begin{tikzpicture}[alt={tikz graphic that works natively that floats}]
    \draw[gray, thick] (-1,2) -- (2,-4);
    \draw[gray, thick] (-1,-1) -- (2,2);
    \filldraw[black] (0,0) circle (2pt) node[anchor=west]{A Tikz Picture that floats! };
  \end{tikzpicture}
  \caption{A caption for a tikz picture}
\end{figure}

\begin{algorithm}
  \caption{Score Algorithm}
  \begin{algorithmic}[1]
    \State {\textbf{Input: }{$s$ is a sensor }}
    \Statex
    \For{$j\in \{1,2,\ldots,15\}$}
    \State Randomly choose $5$ days
    \For{$x\in \{1,2,\ldots,1000\}$}
    \parstate{Set $a$ to be something in this very long state that will have to be wrapped quite possibly around and around and around}
    \EndFor
    \EndFor
  \end{algorithmic}
\end{algorithm}

More text here. Now what is we ?
\autoref{WhatAGreatTheorem}

\printbibliography[heading=subbibnumbered]

