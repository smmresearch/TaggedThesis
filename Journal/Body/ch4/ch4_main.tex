\chapter{\MakeUppercase{Paper 3 Title Goes Here}}
% \MakeUppercase{} is conveninent, but if it causes undesired
%behavior (perhaps due to acronyms or math), you can just manually capitalize your chapter title
% \chaper{GENERAL INTRODUCTION}

\begin{center}
    Authors and Affiliations \\
    Modified from a manuscript to be submitted to/ under review/ published in \textit{Name of the Journal}
    \blfootnote{A version of this chapter appears in Journal of Discipline, Volume 18, Issue 3}
\end{center}

\section{Abstract}
This is the text of my abstract that is part of the thesis itself.
The abstract describes the work in the first paper general. You can use the same abstract as your paper here.
I want to also refer you to \autoref{methods} for more about the procedures.
%\pagebreak %remove if needed

% Please include sections as the paper has, some of the following sections are meant as examples of what can be done, the bibliography should be made as given

\section{Methods and procedures}
\label{methods}

This is the opening paragraph to my thesis which
explains in general terms the concepts and hypothesis
which will be used in my thesis.

With more general information given here than really
necessary.

\section{Introduction}

Here initial concepts and conditions are explained and
several hypothesis are mentioned in brief.

As can be seen in \autoref{nothing} it is truly
obvious what I am saying is true.

\begin{table}[h!tb] \centering
    \caption[This table shows a standard non-empty table. Please check the code caption for extended instructions]{This table shows a standard empty table. In case of long captions, we want to use the long caption as the description to the table and image but not use it in the table of contents and list of figures/ tables. In order to do this, there are two captions which have been provided, remove the first square bracket options if there is only one small caption. You can use citations like this to}
    %use \tagpdfsetup{table/tagging=presentation} if and only if the table
    %is for alignment/presentation purposes and is not a real table
    %
    %\tagpdfsetup{table/header-rows={1,2}} would have rows 1 and 2 be header rows
    \tagpdfsetup{table/header-rows={1}}
    %Use \tagpdfsetup{table/header-columns={}} for header columns instead
    %Put \tagpdfsetup{table/multirow={⟨number of rows⟩}} in cells spanning multiple rows
    %Note, multicolumn works automatically (no extra commands necessary)

    \begin{tabular}{ll}
        Bach      & Cello Suite Number 1  \\
        Beethoven & Cello Sonata Number 3 \\
        Brahms    & Cello Sonata Number 1
    \end{tabular}
    \label{nothing}

    \vspace{ 2 in}
\end{table}

\subsection{Hypothesis}

Here one particular hypothesis is explained in depth
and is examined in the light of current literature.

This can also be seen in \autoref{moon} that the
rest is obvious.

\begin{figure}[h!tb] \centering

    \vspace{ 2 in}
    \caption{This table shows a standard empty figure}
    \label{moon}
\end{figure}

\subsubsection{Parts of the hypothesis}

Here one particular part of the hypothesis that is
currently being explained is examined and particular
elements of that part are given careful scrutiny.

% Below \subsubsection
% Sectional commands: \paragraph and \subparagraph may also be used

\subsection{Second Hypothesis}

Here one particular hypothesis is explained in depth
and is examined in the light of current literature.

\subsubsection{Parts of the second hypothesis}

Here one particular part of the hypothesis that is
currently being explained is examined and particular
elements of that part are given careful scrutiny.

%\addtocontents{toc}{\protect\newpage} % Adds \newpage in "\tableofcontents"
\section{Criteria Review}

Here certain criteria are explained thus eventually
leading to a foregone conclusion as can be seen in
\autoref{nevermore}.

\begin{table}[h!tb] \centering
    \captionsetup{width=2in}
    \caption{This table shows a standard empty table with a limited caption width}
    \label{nevermore}

    \vspace{ 2 in}
\end{table}
\section{Continuing Tables}
Note, tables with cells spanning multiple columns work automatically, but cells spanning multiple rows require extra tagging.
\begin{table}[ht]
    \caption{This is a two-part table doing things.}
    %
    %use \tagpdfsetup{table/tagging=presentation} if and only if the table
    %is for alignment/presentation purposes and is not a real table
    %
    %\tagpdfsetup{table/header-rows={1,2}} would have rows 1 and 2 be header rows
    \tagpdfsetup{table/header-rows={1}}
    %Use \tagpdfsetup{table/header-columns={}} for header columns instead
    %Put \tagpdfsetup{table/multirow={⟨number of rows⟩}} in cells spanning multiple rows
    %Note, multicolumn works automatically (no extra commands necessary)

    \begin{tabular}{rrrlrrrr}
        k      & q          & p+ & p- & s1                                                         & s2                                                         & s3                                                         & RHS       \\ \hline
        2      & 2          & 2  & 1  & 1                                                          & 0                                                          & 0                                                          & 1         \\
        -T     & 0          & 1  & 1  & 0                                                          & 1                                                          & 0                                                          & 0         \\
        T      & -1         & 0  & 1  & 0                                                          & 0                                                          & 1                                                          & 0         \\
        -1     & 1          & -1 & 1  &                                                            &                                                            &                                                            &           \\ \hline
        2(T+1) & 2          & 0  & 1  & 1                                                          & -2                                                         & 0                                                          & 1         \\ \hline
        -T     & 0          & 1  & 1  & 0                                                          & 1                                                          & 0                                                          & 0         \\
        T      & -1         & 0  & 1  & 0                                                          & 0                                                          & 1                                                          & 0         \\
        -(T+1) & 1          & 0  & 1  & 0                                                          & 1                                                          & 0                                                          &           \\ \hline
        0      & 2+2(T+1)/T & 0  & 1  & 1                                                          & -2                                                         & -2(T+1)/T                                                  & 1         \\ \hline
        0      & -1         & 1  & 1  & 0                                                          & 1                                                          & 1                                                          & 0         \\
        1      & -1/T       & 0  & 1  & 0                                                          & 0                                                          & 1/T                                                        & 0         \\
        0      & 1-(T+1)/T  & 0  & 1  & 0                                                          & 1                                                          & (T+1)/T                                                    &           \\ \hline
        0      & 2(2T+1)/T  & 0  & 1  & 1                                                          & -2                                                         & -2(T+1)/T                                                  & 1         \\ \hline
        0      & -1         & 1  & 1  & 0                                                          & 1                                                          & 1                                                          & 0         \\
        1      & -1/T       & 0  & 1  & 0                                                          & 0                                                          & 1/T                                                        & 0         \\
        0      & -1/T       & 0  & 1  & 0                                                          & 1                                                          & (T+1)/T                                                    &           \\ \hline
        0      & 1          & 0  & 1  & T/2(2T+1)                                                  & -T/(2T+1)                                                  & -1                                                         & T/2(2T+1) \\ \hline
        0      & 0          & 1  & 1  & T/2(2T+1)                                                  & 1-T/(2T+1)                                                 & 0                                                          & T/2(2T+1) \\
        1      & 0          & 0  & 1  & 1/2(2T+1)                                                  & -1/(2T+1)                                                  & 0                                                          & 1/2(2T+1) \\ \hline
        0      & 0          & 0  & 1  & 1/2(2T+1)                                                  & 1-1/(2T+1)                                                 & -1+(T+1)/TT                                                &           \\ \hline
        0      & 0          & 0  & 1  & 1/2(2T+1)                                                  & 1-1/(2T+1)                                                 & -1+(T+1)/TT                                                &           \\ \hline
        0      & 0          & 0  & 0  & \tagpdfsetup{table/multirow={8}}\multirow{8}{*}{1/2(2T+1)} & \tagpdfsetup{table/multirow={8}}\multirow{8}{*}{1/2(2T+1)} & \tagpdfsetup{table/multirow={8}}\multirow{8}{*}{1/2(2T+1)} &           \\ \cline{1-4}
        0      & 0          & 0  & 0  &                                                            &                                                            &                                                            &           \\ \cline{1-4}
        0      & 0          & 0  & 0  &                                                            &                                                            &                                                            &           \\ \cline{1-4}
        0      & 0          & 0  & 0  &                                                            &                                                            &                                                            &           \\ \cline{1-4}
        0      & 0          & 0  & 0  &                                                            &                                                            &                                                            &           \\ \cline{1-4}
        0      & 0          & 0  & 0  &                                                            &                                                            &                                                            &           \\ \cline{1-4}
        0      & 0          & 0  & 0  &                                                            &                                                            &                                                            &           \\ \cline{1-4}
        0      & 0          & 0  & 0  &                                                            &                                                            &                                                            &           \\ \hline
    \end{tabular}
\end{table}
\begin{table}[ht]
    %As a continued table, use \ContinuedFloat to keep the same #
    %And use the text "Continued" inside the caption
    \ContinuedFloat
    \caption{Continued}
    %
    %use \tagpdfsetup{table/tagging=presentation} if and only if the table
    %is for alignment/presentation purposes and is not a real table
    %
    %\tagpdfsetup{table/header-rows={1,2}} would have rows 1 and 2 be header rows
    \tagpdfsetup{table/header-rows={1}}
    %Use \tagpdfsetup{table/header-columns={}} for header columns instead
    %Put \tagpdfsetup{table/multirow={⟨number of rows⟩}} in cells spanning multiple rows
    %Note, multicolumn works automatically (no extra commands necessary)

    \begin{tabular}{rrrlrrrr}
        k      & q          & p+ & \multicolumn{1}{r}{p-} & s1        & s2         & s3          & RHS       \\ \hline
        2      & 2          & 2  & \multicolumn{1}{r}{-2} & 1         & 0          & 0           & 1         \\
        -T     & 0          & 1  & -1                     & 0         & 1          & 0           & 0         \\
        T      & -1         & 0  & 0                      & 0         & 0          & 1           & 0         \\
        -1     & 1          & -1 & \multicolumn{1}{r}{1}  &           &            &             &           \\ \hline
        2(T+1) & 2          & 0  & 0                      & 1         & -2         & 0           & 1         \\ \hline
        -T     & 0          & 1  & -1                     & 0         & 1          & 0           & 0         \\
        T      & -1         & 0  & 0                      & 0         & 0          & 1           & 0         \\
        -(T+1) & 1          & 0  & 0                      & 0         & 1          & 0           &           \\ \hline
        0      & 2+2(T+1)/T & 0  & 0                      & 1         & -2         & -2(T+1)/T   & 1         \\ \hline
        0      & -1         & 1  & \multicolumn{1}{r}{-1} & 0         & 1          & 1           & 0         \\
        1      & -1/T       & 0  & 0                      & 0         & 0          & 1/T         & 0         \\
        0      & 1-(T+1)/T  & 0  & 0                      & 0         & 1          & (T+1)/T     &           \\ \hline
        0      & 2(2T+1)/T  & 0  & 0                      & 1         & -2         & -2(T+1)/T   & 1         \\ \hline
        0      & -1         & 1  & \multicolumn{1}{r}{-1} & 0         & 1          & 1           & 0         \\
        1      & -1/T       & 0  & 0                      & 0         & 0          & 1/T         & 0         \\
        0      & -1/T       & 0  & 0                      & 0         & 1          & (T+1)/T     &           \\ \hline
        0      & 1          & 0  & \multicolumn{1}{r}{0}  & T/2(2T+1) & -T/(2T+1)  & -1          & T/2(2T+1) \\ \hline
        0      & 0          & 1  & \multicolumn{1}{r}{-1} & T/2(2T+1) & 1-T/(2T+1) & 0           & T/2(2T+1) \\
        1      & 0          & 0  & \multicolumn{1}{r}{0}  & 1/2(2T+1) & -1/(2T+1)  & 0           & 1/2(2T+1) \\ \hline
        0      & 0          & 0  & 0                      & 1/2(2T+1) & 1-1/(2T+1) & -1+(T+1)/TT &           \\ \hline
    \end{tabular}
\end{table}

\section{Results}
Include any results

\section{Conclusion}\label{conclusion2}

The conclusion of the paper goes here.

%\autocite{allen}, \autocite{bruner} 
\autocite{dochtermannMinimalGraphsContractible2023}
\autocite{superLong}
%%%%
% Reference section comes before the appendix

\printbibliography[heading=subbibnumbered]

%%%%%
%% Appendix
% This section may or may not be included
% Chapter 2 shows a double appendix example. Please use A or B as in Appendix A if there are multiple appendix. Use "Appendix A:" before writing the title
% Chapter 3 shows a single appendix example. Use "Appendix:" before writing the title 
\section{Appendix: Appendix Title Goes Here}
If there is an appendix that needs to go with the paper it can be as a section \autocite{virkContractibilityRipsComplexes2024}

\subsection{Procedure details}
Details of the paper specific appendix procedures