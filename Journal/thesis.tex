%!LW recipe=latexmk (lualatex)
%The above command is for use with VSCode as an editor - can be deleted for others

%%%%%%% Instructions %%%%%%%%%%%%%%%%%%%

%Use lualatex to compile with --interaction-mode=nonstopmode
%The commands I run are of the form (in the folder containing thesis.tex): 

%lualatex --aux-directory=./LatexAux --synctex=1 --interaction=nonstopmode --c-style-errors .\thesis.tex
%biber thesis --output-directory=.\LatexAux  

%biber's output-directory should match lualatex's aux-directory or both should be omitted. c-style-errors is optional.
%You can also use latexmk if desired. Beware, sometimes if you have just added alot or are starting fresh,
%it can take more than 5 runs (latexmk's default max) for everything to settle.


%Therefore, do not use amssymb, input/output enc
% Get Error ".. \@ doesn't match definition.."? Try deleting *.aux file of file you were messing with. If needed, escalate to deleting all auxillary files.

% Did you know you can have the auxillary files in a different directory with MikTeX?
% Makes deleting all of them a snap

%%Using Tikz? Externally generate as pdf (with just normal presets/without tagging) and use includegraphics to add alt text



%% Tagging is slow
%%
%% Things that can be commented or uncommented in this tex file
%% while drafting
%% (and should be un-done before submitting)
%% to speed-up compliation
%% 
%% 1. Use "\AddToHook{enddocument/end}{\tagpdfsetup{activate/tree=false}}"
    %Find this commented out above DocumentMetadata
    %This command disables a slow part of tagging
    %May want to disable periodically to ensure tagging is working
    %Comment out before submission
%% 2. Use "\tagpdfsetup{
              % math/mathml/sources=,
              % math/mathml/luamml/load=false,
              % math/tex/AF=false
              % }"
    %Find this command commented out below DocumentMetadata
    %This command disables math related accessiblity tagging (which is also slow)
    %Math tagging is not required by Iowa State, but is nice to have.
%%%%%%%%%%%%%%%%%%%%%%%%%%%%%%%%%%%%%%%%%%%%%%%%%%%%%%%%%%%%%%%%%%%%%%%%%%%%%%%%%%%%%%%%%%%%%%%%%%%%%%%%%%%%%%%%%%%%%%%%%


%Command below disables slow parts of tagging,
%speeding up compiling during drafting stages
%When submitting, it should be commented out
%\AddToHook{enddocument/end}{\tagpdfsetup{activate/tree=false}}
%

\DocumentMetadata{
 lang=en,
 testphase={
  phase-III,
  math,
  table,
  title,
  firstaid
  },
 pdfversion=2.0,
 pdfstandard=ua-2,
 pdfstandard=a-4f,
 uncompress
}

%The command below disables the slow parts of tagging math, 
%speeding up compiling
% \tagpdfsetup{
%  math/mathml/sources=,
%  math/mathml/luamml/load=false,
%  math/tex/AF=false
% }

% Template file for a standard thesis
\documentclass[11pt,notitlepage]{isuthesistagged}
% notitlepage is used because \begin{abstract} uses titlepage by default, which resets the page numbers


%Caption package & tagging fixes
\usepackage{caption}
\usepackage{subcaption}
%This work-around comes from: https://github.com/latex3/tagging-project/issues/720
\RemoveFromHook{begindocument}[latex-lab-testphase-float]
\makeatletter
\ExplSyntaxOn
\socket_new_plug:nnn{tagsupport/parbox/before}{caption}
  {   
   \tagpdfparaOn %restart para tagging
   \tl_if_empty:NTF\@current@float@struct
    {     
     \tag_struct_begin:n{tag=Caption,firstkid}
    }
    {
     \tag_struct_begin:n{tag=Caption,parent=\@current@float@struct,firstkid}
    } 
  }
\socket_new_plug:nnn{tagsupport/parbox/after}{caption}
  {
   \tag_struct_end:   
  }
\l@addto@macro\caption@beginex@hook{%
  \tagpdfparaOff %leavevmode in parbox should not start paragraph structure
  \AssignSocketPlug{tagsupport/parbox/before}{caption}
  \AssignSocketPlug{tagsupport/parbox/after}{caption}}
\ExplSyntaxOff
\AtBeginDocument{
	\renewcommand*\caption@anchor[1]{%
        \ifmeasuring@ \else
           \caption@raisedlink{\MakeLinkTarget*{#1}}%
        \fi}%  
}

%This part is my own invention - to get subfigures
%to have captions be where they are supposed to be
%by appending some extras to the subfigure environment
%As with everything else, it probably won't be necessary
%in the long term
%
%Further, no warranties on the robustness
%If complicated subfigures are causing problems,
%you are probably are better commenting this out
%and dealing with the subpar tagging.
\ExplSyntaxOn
\cs_new_eq:NN \origsubfigure \subfigure 
\cs_new_eq:NN \origsubfigureend \endsubfigure
\RenewDocumentCommand{\subfigure}{O{b} O{} O{s} m}{
  \origsubfigure[#1][#2][#3]{#4}
  \tagstructbegin{tag=Part}
  \tl_set:Ne\@current@float@struct{\tag_get:n{struct_num}}
}
\ExplSyntaxOff
\def\endsubfigure{
  \tagstructend
  \origsubfigureend
}
\makeatother

%In the tagging structure, this collects the floats in each
%section. Without it, all the floats for the entire document
%are collected.
\AddToHook{cmd/section/before}{\tagtool{flush-floats=subsection}}

%For tables
\usepackage{multirow}

%%Mathy packages
\usepackage{amsmath}
\usepackage{amsthm}
\usepackage{mathtools}
\usepackage{thmtools}
\usepackage{templatedShortcuts-private}
%algorithm packages
\usepackage[chapter]{algorithm}
\usepackage[noend]{algpseudocode}
\makeatletter
\newcommand{\algmargin}{\the\ALG@thistlm}
\makeatother
\renewcommand{\thealgorithm}{\arabic{chapter}.\arabic{algorithm}} 
\algnewcommand{\parstate}[1]{\State%
  \parbox[t]{\dimexpr\linewidth-\algmargin}{\strut #1\strut}}
%Following makes proof more reliable for tagging, from: https://github.com/latex3/tagging-project/issues/790
\makeatletter
\renewenvironment{proof}[1][\proofname]{\par
  \pushQED{\qed}%
  \normalfont \topsep6\p@\@plus6\p@\relax
  \trivlist
  \item[\hskip\labelsep
        \itshape
    #1\@addpunct{.}]\ignorespaces
}{%
  \popQED\endtrivlist\par\@endpefalse %<-- par inserted
}
\makeatother

%%

%Check package documentation for other font options
%For example, the "default" is a Computer Modern clone
\usepackage[stixtwo]{fontsetup}

%Instead of font setup, unicode-math has more fine-tuned control
%\usepackage{unicode-math}
% \setmainfont{texgyrepagella}[
% Extension       = .otf,
% UprightFont     = *-regular,
% ItalicFont      = *-italic,
% BoldFont        = *-bold,
% BoldItalicFont  = *-bolditalic 
% ]
% \setmathfont{texgyrepagella-math.otf}


\usepackage{graphicx}
\chaptertitle
% Old-style, thesis numbering down to subsubsection
\alternate
%Natbib compatibility mode activated
%Different styles available - look at biblatex documentation
\usepackage[natbib=true,refsection=chapter,style=authoryear]{biblatex}
\addbibresource{thesisAccessTest.bib}
\setlength{\bibitemsep}{13.2pt}

%Set the author
\newcommand{\theAuthor}{Alice Wonder}
%Set the tite 
\twoLineTitle{This is the title of a thesis
submitted to Iowa State University on the first line.}{
Note that only the first letter of
the first word and proper names
are capitalized and this is the second line.}

%%PDF properties automatically set
\author{\theAuthor}
\title{\titleWithLineBreak}
\usepackage[hypertexnames=false,linktocpage=true,pdfauthor={\theAuthor},pdftitle={\titleWithoutBreak},]{hyperref}
\hypersetup{colorlinks=true,linkcolor=blue,citecolor=blue,filecolor=blue,urlcolor=blue,bookmarksnumbered=true,pdfview=FitB,pdfencoding=auto}
\usepackage{bookmark}

\renewcommand{\chapterautorefname}{Chapter}
\renewcommand{\sectionautorefname}{Section}
\renewcommand{\subsectionautorefname}{Subsection}

\overfullrule=0pt
%%%%%%%%%%%%%%%%%%%%%

\usepackage{etoolbox}
\makeatletter
\setlength{\@fptop}{0pt} %command added to ensure images always float on top of the page
\makeatother

\usepackage{xcolor}
\usepackage{geometry}
\geometry{letterpaper, left=1in, top=1in, right=1in, bottom=1in, includehead=true,headheight=14pt} 
\usepackage{pdflscape}
%%%%%%%%%%%%%%%%%%%%%%%%%%%%%%%%%%

\usepackage[notintoc,english]{nomencl}

\doublespacing
\AtBeginEnvironment{table}{\singlespacing}% 
\AtBeginEnvironment{figure}{\singlespacing}% 

%%%Captioning Format
\DeclareCaptionListFormat{figureList}{Figure #1#2}
\DeclareCaptionListFormat{tableList}{Table #1#2}
\captionsetup[figure]{listformat=figureList}
\captionsetup[table]{listformat=tableList}
\captionsetup[ContinuedFloat]{list=no}

%The command below can be used to get rid of footnote rule(line)
%\renewcommand*\footnoterule{}

%With unicode instead of bm, symbf should be used
%This remaps \bm to prevent problems. There might be slightly different behavior.
\let\bm\symbf

\begin{document}
\DeclareGraphicsExtensions{.jpg,.pdf,.mps,.png}
%command used in preface
\def\@makechapterheada{\vspace*{-2cm}\titlepage}
% Template Titlepage File
% Please choose appropriate options for Master's thesis, Doctoral dissertations, and creative components. Please read the comments to make an informed choice

\vspace*{-2cm}\titlepage
%%%%%%%%%%%%%%%%%%%

\degree{DOCTOR OF PHILOSOPHY}
\major{Mathematics}
%For co-majors use this command instead of major. Separate the co-majors with a semi-colon
%\comajors{Statistics; Computer Science}

\mprof{John Smith}
% If you have co-major professors, use \mprofs for the first, and \cmprofs for the second 
% instead of mprof
%\mprofs{ABC}
%\cmprofs{DEF}

\format{dissertation}%change to thesis or creative component as needed for a master's
\members{Jane Dee \\ Allen Wrench\\ Katniss Everdeen}
\disclaimertitlepage %sets-up title page to add disclaimer at end
\notice %sets-up title page to add the copyright notice at the bottom

%%%%%%
% "A <format> submitted to <the graduate faculty>..." can
% be changed to something other than graduate faculty with
% this command:
% \submit{the graduate faculty alternative}

\maketitle

\fancypagestyle{plain}{}

\raggedright
\parindent 0.25 in % set all paragraphs in the document to have indent

% %%%%%%%%%%%%%%%%%%%%%%%%%%%%%%%%%%%%%%%%%
% %% The line below adds the word "Page" over the page numbers in TOC, LOT, LOF
\addtocontents{toc}{~\hfill\tagstructbegin{tag=Lbl,stash,label=pageOfTOC}\tagmcbegin{tag=Lbl}\textbf{Page}\par\tagmcend\tagstructend}
\addtocontents{lot}{~\hfill\tagstructbegin{tag=Lbl,stash,label=pageOfLOT}\tagmcbegin{tag=Lbl}\textbf{Page}\par\tagmcend\tagstructend}
\addtocontents{lof}{~\hfill\tagstructbegin{tag=Lbl,stash,label=pageOfLOF}\tagmcbegin{tag=Lbl}\textbf{Page}\par\tagmcend\tagstructend}
% %%

% %%
% % Optional thesis dedication
%   Dedication is not usually listed in the table of contents.
%   However, if you do want it, add this command here (not in the dedication file)
%   \addToTOCWithoutChapter{DEDICATION}
\include{Preface/dedication}

{
  \pdfbookmark[0]{TABLE OF CONTENTS}{table}
  \tableofcontents
  \tagstructuse{pageOfTOC} %gets tagging of "Page" done
}
\addtocontents{toc}{\def\protect\@chapapp{}} \cleardoublepage \phantomsection
\pagebreak
% %%%%%%%%%%%%%%%%%%%%%%%%%%%%%%%%%%%%%%%%%
\MakeLinkTarget[specialchapter]{}%necessary for tagging (as of 2024)
\addcontentsline{toc}{chapter}{LIST OF TABLES}
\listoftables
\tagstructuse{pageOfLOT}%gets tagging of "Page" done
% %%%%%%%%%%%%%%%%%%%%%%%%%%%%%%%%%%%%%%%%%
\cleardoublepage \phantomsection
\MakeLinkTarget[specialchapter]{} %necessary for tagging (as of 2024)
\addcontentsline{toc}{chapter}{LIST OF FIGURES}
\listoffigures
\tagstructuse{pageOfLOF}%gets tagging of "Page" done
% %%%%%%%%%%%%%%%%%%%%%%%%%%%%%%%%%%%%%%%%%

%Optional nomenclature
% \cleardoublepage \phantomsection
%\MakeLinkTarget[specialchapter]{}
% \makenomenclature
\renewcommand{\nomname}{NOMENCLATURE}
%\specialchapt{NOMENCLATURE}

%\mbox{}
\renewcommand\nomgroup[1]{%
  \item[\bfseries
  \ifstrequal{#1}{A}{Physics Constants}{%
  \ifstrequal{#1}{B}{Number Sets}{%
  \ifstrequal{#1}{C}{Other Symbols}{}}}%
]}

\nomenclature[A, 02]{$c$}{Speed of light in a vacuum inertial system}
\nomenclature[A, 03]{$h$}{Plank Constant}
\nomenclature[A, 01]{$g$}{Gravitational Constant}
\nomenclature[B, 03]{$\mathbb{R}$}{Real Numbers}
\nomenclature[B, 02]{$\mathbb{C}$}{Complex Numbers}
\nomenclature[B, 01]{$\mathbb{H}$}{Octonions}
\nomenclature[C]{$V$}{Constant Volume}
\nomenclature[C]{$\rho$}{Friction Index}

\renewcommand{\nompreamble}{The nomenclature for your dissertation or thesis is optional. This list may be placed in
the following places: as the last preliminary page, before the Reference section, or as an Appendix. The heading is bold if other major headings are bold, and the list is in the same font size and style as text. Nomenclature should follow a two-column format with the term in the left
column and its definition or description within the right column.}

\printnomenclature

% The following link has more tweaks, tips and tricks on how to setup nomenclatures: https://www.overleaf.com/learn/latex/Nomenclatures

%Adds Chapter in front of chapter on TOC
\addtocontents{toc}{\def\protect\@chapapp{CHAPTER\ }}

%Optional Acknowledgements
\cleardoublepage \phantomsection
\include{Preface/acknowl}
%Optional thesis abstract
\cleardoublepage \phantomsection
\include{Preface/abstract}
\newpage
\pagenumbering{arabic}
\pagestyle{fancy}
% Chapter 1 of the Thesis Template File
\chapter{GENERAL INTRODUCTION}
This chapter will have the introduction to your thesis as a whole.

This is the opening paragraph to my thesis which
explains in general terms the concepts and hypothesis
which will be used in my thesis.

With more general information given here than really
necessary.

\section{Overview Two Words}

Here initial concepts and conditions are explained and
several hypothesis are mentioned in brief.

\subsection{Hypothesis}

Here one particular hypothesis is explained in depth
and is examined in the light of current literature.

\subsubsection{Parts of the hypothesis}

Here one particular part of the hypothesis that is
currently being explained is examined and particular
elements of that part are given careful scrutiny.

% % Below \subsubsection
% % Sectional commands: \paragraph and \subparagraph may also be used

\subsection{Second Hypothesis}

Here one particular hypothesis is explained in depth
and is examined in the light of current literature.

\subsubsection{Parts of the second hypothesis}

Here one particular part of the hypothesis that is
currently being explained is examined and particular
elements of that part are given careful scrutiny
\autocite{buiEveryGeneratingPolytope2023}, abcd.

\section{Criteria Review}

Here certain criteria are explained thus eventually
leading to a foregone conclusion.
\begin{theorem}
    Here's a theorem!
\end{theorem}

\printbibliography[heading=subbibnumbered]

\chapter{PAPER 1 TITLE GOES HERE}
\label{polymer_fibers}

\begin{center}
    Authors and Affiliations \\
    %The changes suggest this is supposed to be a footnote ... so we are just guessing over here
    Modified from a manuscript to be submitted to/ under review/ published in \textit{Name of the Journal}
    \blfootnote{A version of this chapter appears in Journal of Discipline, Volume 18, Issue 3}
\end{center}

\section{Abstract}
This is the text of my abstract that is part of the thesis itself.
The abstract describes the work in the first paper general. You can use the same abstract as your paper here.

%\pagebreak %remove if needed

% Please include sections as the paper has, some of the following sections are meant as examples of what can be done, the bibliography should be made as given

\section{Overview}

The  construct of this section or any further section is same as the authors paper.
This is the opening paragraph to my thesis which
explains in general terms the concepts and hypothesis
which will be used in my thesis.

With more general information given here than really
necessary.

\section{Introduction}

Here initial concepts and conditions are explained and
several hypothesis are mentioned in brief.

\autocite{kleeHellyTheoremIts1963} the definitive model is seen.

\subsection{Hypothesis}

Here one particular hypothesis is explained in depth
and is examined in the light of current literature.

\subsubsection{Parts of the hypothesis}

Here one particular part of the hypothesis that is
currently being explained is examined and particular
elements of that part are given careful scrutiny.

% Below \subsubsection
% Sectional commands: \paragraph and \subparagraph may also be used

\subsection{Second Hypothesis}

\paragraph{Heading} Here one particular hypothesis is explained in depth
and is examined in the light of current literature. \subparagraph{Even smaller heading} Another sentence.

\subsubsection{Parts of the second hypothesis}

Here one particular part of the hypothesis that is
currently being explained is examined and particular
elements of that part are given careful scrutiny.
\begin{theorem}
    If true, then this theorem is vacuous.
\end{theorem}

\section{Criteria Review}

Here certain criteria are explained thus eventually
leading to a foregone conclusion.

\section{Conclusion}\label{conclusion}

The conclusion of the paper goes here.
%%%%%%%%
\autocite{buiEveryGeneratingPolytope2023}
%%%%
% Reference section comes before the appendix

\printbibliography[heading=subbibnumbered]

%%%%%
%% Appendix
% This section may or may not be included
% Chapter 2 shows a double appendix example. Please use A or B as in Appendix A if there are multiple appendix. Use "Appendix A:" before writing the title
% Chapter 3 shows a single appendix example. Use "Appendix:" before writing the title 
\section{Appendix A: Appendix A Title Goes Here After The Colon}
If there is an appendix that needs to go with the paper it can be as a section \autocite{kleeHellyTheoremIts1963}

\subsection{Procedure details}
Details of the paper specific appendix procedures

% This section may or may not be included

\section{Appendix B: Appendix B Title Goes Here After The Colon}
If there is an appendix that needs to go with the paper it can be as a section \autocite{chenGraphHomotopyGraham2001}

\subsection{Procedure details}
Details of the paper specific appendix procedures
\include{Body/ch3/ch3_main}
\part{Let us have a part page}
\chapter{\MakeUppercase{Paper 3 Title Goes Here}}
% \MakeUppercase{} is conveninent, but if it causes undesired
%behavior (perhaps due to acronyms or math), you can just manually capitalize your chapter title
% \chaper{GENERAL INTRODUCTION}

\begin{center}
    Authors and Affiliations \\
    Modified from a manuscript to be submitted to/ under review/ published in \textit{Name of the Journal}
    \blfootnote{A version of this chapter appears in Journal of Discipline, Volume 18, Issue 3}
\end{center}

\section{Abstract}
This is the text of my abstract that is part of the thesis itself.
The abstract describes the work in the first paper general. You can use the same abstract as your paper here.
I want to also refer you to \autoref{methods} for more about the procedures.
%\pagebreak %remove if needed

% Please include sections as the paper has, some of the following sections are meant as examples of what can be done, the bibliography should be made as given

\section{Methods and procedures}
\label{methods}

This is the opening paragraph to my thesis which
explains in general terms the concepts and hypothesis
which will be used in my thesis.

With more general information given here than really
necessary.

\section{Introduction}

Here initial concepts and conditions are explained and
several hypothesis are mentioned in brief.

As can be seen in \autoref{nothing} it is truly
obvious what I am saying is true.

\begin{table}[h!tb] \centering
    \caption[This table shows a standard non-empty table. Please check the code caption for extended instructions]{This table shows a standard empty table. In case of long captions, we want to use the long caption as the description to the table and image but not use it in the table of contents and list of figures/ tables. In order to do this, there are two captions which have been provided, remove the first square bracket options if there is only one small caption. You can use citations like this to}
    %use \tagpdfsetup{table/tagging=presentation} if and only if the table
    %is for alignment/presentation purposes and is not a real table
    %
    %\tagpdfsetup{table/header-rows={1,2}} would have rows 1 and 2 be header rows
    \tagpdfsetup{table/header-rows={1}}
    %Use \tagpdfsetup{table/header-columns={}} for header columns instead
    %Put \tagpdfsetup{table/multirow={⟨number of rows⟩}} in cells spanning multiple rows
    %Note, multicolumn works automatically (no extra commands necessary)

    \begin{tabular}{ll}
        Bach      & Cello Suite Number 1  \\
        Beethoven & Cello Sonata Number 3 \\
        Brahms    & Cello Sonata Number 1
    \end{tabular}
    \label{nothing}

    \vspace{ 2 in}
\end{table}

\subsection{Hypothesis}

Here one particular hypothesis is explained in depth
and is examined in the light of current literature.

This can also be seen in \autoref{moon} that the
rest is obvious.

\begin{figure}[h!tb] \centering

    \vspace{ 2 in}
    \caption{This table shows a standard empty figure}
    \label{moon}
\end{figure}

\subsubsection{Parts of the hypothesis}

Here one particular part of the hypothesis that is
currently being explained is examined and particular
elements of that part are given careful scrutiny.

% Below \subsubsection
% Sectional commands: \paragraph and \subparagraph may also be used

\subsection{Second Hypothesis}

Here one particular hypothesis is explained in depth
and is examined in the light of current literature.

\subsubsection{Parts of the second hypothesis}

Here one particular part of the hypothesis that is
currently being explained is examined and particular
elements of that part are given careful scrutiny.

%\addtocontents{toc}{\protect\newpage} % Adds \newpage in "\tableofcontents"
\section{Criteria Review}

Here certain criteria are explained thus eventually
leading to a foregone conclusion as can be seen in
\autoref{nevermore}.

\begin{table}[h!tb] \centering
    \captionsetup{width=2in}
    \caption{This table shows a standard empty table with a limited caption width}
    \label{nevermore}

    \vspace{ 2 in}
\end{table}
\section{Continuing Tables}
Note, tables with cells spanning multiple columns work automatically, but cells spanning multiple rows require extra tagging.
\begin{table}[ht]
    \caption{This is a two-part table doing things.}
    %
    %use \tagpdfsetup{table/tagging=presentation} if and only if the table
    %is for alignment/presentation purposes and is not a real table
    %
    %\tagpdfsetup{table/header-rows={1,2}} would have rows 1 and 2 be header rows
    \tagpdfsetup{table/header-rows={1}}
    %Use \tagpdfsetup{table/header-columns={}} for header columns instead
    %Put \tagpdfsetup{table/multirow={⟨number of rows⟩}} in cells spanning multiple rows
    %Note, multicolumn works automatically (no extra commands necessary)

    \begin{tabular}{rrrlrrrr}
        k      & q          & p+ & p- & s1                                                         & s2                                                         & s3                                                         & RHS       \\ \hline
        2      & 2          & 2  & 1  & 1                                                          & 0                                                          & 0                                                          & 1         \\
        -T     & 0          & 1  & 1  & 0                                                          & 1                                                          & 0                                                          & 0         \\
        T      & -1         & 0  & 1  & 0                                                          & 0                                                          & 1                                                          & 0         \\
        -1     & 1          & -1 & 1  &                                                            &                                                            &                                                            &           \\ \hline
        2(T+1) & 2          & 0  & 1  & 1                                                          & -2                                                         & 0                                                          & 1         \\ \hline
        -T     & 0          & 1  & 1  & 0                                                          & 1                                                          & 0                                                          & 0         \\
        T      & -1         & 0  & 1  & 0                                                          & 0                                                          & 1                                                          & 0         \\
        -(T+1) & 1          & 0  & 1  & 0                                                          & 1                                                          & 0                                                          &           \\ \hline
        0      & 2+2(T+1)/T & 0  & 1  & 1                                                          & -2                                                         & -2(T+1)/T                                                  & 1         \\ \hline
        0      & -1         & 1  & 1  & 0                                                          & 1                                                          & 1                                                          & 0         \\
        1      & -1/T       & 0  & 1  & 0                                                          & 0                                                          & 1/T                                                        & 0         \\
        0      & 1-(T+1)/T  & 0  & 1  & 0                                                          & 1                                                          & (T+1)/T                                                    &           \\ \hline
        0      & 2(2T+1)/T  & 0  & 1  & 1                                                          & -2                                                         & -2(T+1)/T                                                  & 1         \\ \hline
        0      & -1         & 1  & 1  & 0                                                          & 1                                                          & 1                                                          & 0         \\
        1      & -1/T       & 0  & 1  & 0                                                          & 0                                                          & 1/T                                                        & 0         \\
        0      & -1/T       & 0  & 1  & 0                                                          & 1                                                          & (T+1)/T                                                    &           \\ \hline
        0      & 1          & 0  & 1  & T/2(2T+1)                                                  & -T/(2T+1)                                                  & -1                                                         & T/2(2T+1) \\ \hline
        0      & 0          & 1  & 1  & T/2(2T+1)                                                  & 1-T/(2T+1)                                                 & 0                                                          & T/2(2T+1) \\
        1      & 0          & 0  & 1  & 1/2(2T+1)                                                  & -1/(2T+1)                                                  & 0                                                          & 1/2(2T+1) \\ \hline
        0      & 0          & 0  & 1  & 1/2(2T+1)                                                  & 1-1/(2T+1)                                                 & -1+(T+1)/TT                                                &           \\ \hline
        0      & 0          & 0  & 1  & 1/2(2T+1)                                                  & 1-1/(2T+1)                                                 & -1+(T+1)/TT                                                &           \\ \hline
        0      & 0          & 0  & 0  & \tagpdfsetup{table/multirow={8}}\multirow{8}{*}{1/2(2T+1)} & \tagpdfsetup{table/multirow={8}}\multirow{8}{*}{1/2(2T+1)} & \tagpdfsetup{table/multirow={8}}\multirow{8}{*}{1/2(2T+1)} &           \\ \cline{1-4}
        0      & 0          & 0  & 0  &                                                            &                                                            &                                                            &           \\ \cline{1-4}
        0      & 0          & 0  & 0  &                                                            &                                                            &                                                            &           \\ \cline{1-4}
        0      & 0          & 0  & 0  &                                                            &                                                            &                                                            &           \\ \cline{1-4}
        0      & 0          & 0  & 0  &                                                            &                                                            &                                                            &           \\ \cline{1-4}
        0      & 0          & 0  & 0  &                                                            &                                                            &                                                            &           \\ \cline{1-4}
        0      & 0          & 0  & 0  &                                                            &                                                            &                                                            &           \\ \cline{1-4}
        0      & 0          & 0  & 0  &                                                            &                                                            &                                                            &           \\ \hline
    \end{tabular}
\end{table}
\begin{table}[ht]
    %As a continued table, use \ContinuedFloat to keep the same #
    %And use the text "Continued" inside the caption
    \ContinuedFloat
    \caption{Continued}
    %
    %use \tagpdfsetup{table/tagging=presentation} if and only if the table
    %is for alignment/presentation purposes and is not a real table
    %
    %\tagpdfsetup{table/header-rows={1,2}} would have rows 1 and 2 be header rows
    \tagpdfsetup{table/header-rows={1}}
    %Use \tagpdfsetup{table/header-columns={}} for header columns instead
    %Put \tagpdfsetup{table/multirow={⟨number of rows⟩}} in cells spanning multiple rows
    %Note, multicolumn works automatically (no extra commands necessary)

    \begin{tabular}{rrrlrrrr}
        k      & q          & p+ & \multicolumn{1}{r}{p-} & s1        & s2         & s3          & RHS       \\ \hline
        2      & 2          & 2  & \multicolumn{1}{r}{-2} & 1         & 0          & 0           & 1         \\
        -T     & 0          & 1  & -1                     & 0         & 1          & 0           & 0         \\
        T      & -1         & 0  & 0                      & 0         & 0          & 1           & 0         \\
        -1     & 1          & -1 & \multicolumn{1}{r}{1}  &           &            &             &           \\ \hline
        2(T+1) & 2          & 0  & 0                      & 1         & -2         & 0           & 1         \\ \hline
        -T     & 0          & 1  & -1                     & 0         & 1          & 0           & 0         \\
        T      & -1         & 0  & 0                      & 0         & 0          & 1           & 0         \\
        -(T+1) & 1          & 0  & 0                      & 0         & 1          & 0           &           \\ \hline
        0      & 2+2(T+1)/T & 0  & 0                      & 1         & -2         & -2(T+1)/T   & 1         \\ \hline
        0      & -1         & 1  & \multicolumn{1}{r}{-1} & 0         & 1          & 1           & 0         \\
        1      & -1/T       & 0  & 0                      & 0         & 0          & 1/T         & 0         \\
        0      & 1-(T+1)/T  & 0  & 0                      & 0         & 1          & (T+1)/T     &           \\ \hline
        0      & 2(2T+1)/T  & 0  & 0                      & 1         & -2         & -2(T+1)/T   & 1         \\ \hline
        0      & -1         & 1  & \multicolumn{1}{r}{-1} & 0         & 1          & 1           & 0         \\
        1      & -1/T       & 0  & 0                      & 0         & 0          & 1/T         & 0         \\
        0      & -1/T       & 0  & 0                      & 0         & 1          & (T+1)/T     &           \\ \hline
        0      & 1          & 0  & \multicolumn{1}{r}{0}  & T/2(2T+1) & -T/(2T+1)  & -1          & T/2(2T+1) \\ \hline
        0      & 0          & 1  & \multicolumn{1}{r}{-1} & T/2(2T+1) & 1-T/(2T+1) & 0           & T/2(2T+1) \\
        1      & 0          & 0  & \multicolumn{1}{r}{0}  & 1/2(2T+1) & -1/(2T+1)  & 0           & 1/2(2T+1) \\ \hline
        0      & 0          & 0  & 0                      & 1/2(2T+1) & 1-1/(2T+1) & -1+(T+1)/TT &           \\ \hline
    \end{tabular}
\end{table}

\section{Results}
Include any results

\section{Conclusion}\label{conclusion2}

The conclusion of the paper goes here.

%\autocite{allen}, \autocite{bruner} 
\autocite{dochtermannMinimalGraphsContractible2023}
\autocite{superLong}
%%%%
% Reference section comes before the appendix

\printbibliography[heading=subbibnumbered]

%%%%%
%% Appendix
% This section may or may not be included
% Chapter 2 shows a double appendix example. Please use A or B as in Appendix A if there are multiple appendix. Use "Appendix A:" before writing the title
% Chapter 3 shows a single appendix example. Use "Appendix:" before writing the title 
\section{Appendix: Appendix Title Goes Here}
If there is an appendix that needs to go with the paper it can be as a section \autocite{virkContractibilityRipsComplexes2024}

\subsection{Procedure details}
Details of the paper specific appendix procedures
\include{Body/ch5/ch5_main}
\newcommand{\RipsD}{\operatorname{Rips}_1}
\newcommand{\Rips}{\operatorname{Rips}}
\newcommand{\ver}{\operatorname{ver}}
\newcommand{\diam}{\operatorname{Diam}}
\newcommand{\midR}{\operatorname{mid}}
\newcommand{\dN}{N_1^*}

\chapter{CHAPTER WITH MATH}
\begin{center}
    Authors and Affiliations \\
    Modified from a manuscript to be submitted to/ under review/ published in \textit{Name of the Journal}
    \blfootnote{A version of this chapter appears in Journal of Discipline, Volume 18, Issue 3}
\end{center}

\section{Abstract}
This is the text of my abstract that is part of the thesis itself.
The abstract describes the work in the first paper general. You can use the same abstract as your paper here.
\section{Proofs and Stuff}
\begin{definition}
    A set $A$ is something.
\end{definition}

\begin{lemma}
    If cool, then great.
\end{lemma}
\begin{proof}
    Without loss of generality, it works.
    \begin{equation}
    \label{orthogonalProjectionIsGoodActually}
        d(x,y)= d(x,z)+d(z,y) \geq d(x,x-\langle x,n\rangle n )+0 = \langle x,n\rangle.
    \end{equation}

    Furthermore,
    \begin{align}
        \ell_1(\hat{x},y) & = \ell_1 (x,y)                     \\
                          & =|\ell_1(x,y)-2\langle x,n\rangle \ell_1(n,0)|        
    \end{align}
    From \autoref{orthogonalProjectionIsGoodActually}, it follows \[\ell_1(\hat{x},y)\ \ell_1(\hat{x},y)-2\langle x,n\rangle \leq\] 
\end{proof}




\begin{theorem}
    If true, then it all collapses.
\end{theorem}
\begin{proof}
    By Zorn's lemma, Zorn has the best name \autocite{martiniCompleteReducedConvex2019}.
    Also, \autocite{chenGraphHomotopyGraham2001} and \autocite{dochtermannMinimalGraphsContractible2023}.
    \[x^2+y^2+x^2=2.\]

\end{proof}

\printbibliography[heading=subbibnumbered]


\include{Body/ch6/ch6_future_and_conclusions}
\clearpage
\pagebreak
\end{document}

% IMPORTANT NOTES
% TABLE OF CONTENTS :
% TOPIC 1:  If you need a page break follow the steps below
% step1
% check before which chapter in the table of contents you want a page break
% step 2
% go the folder "body". There open the chapter tex file that you noted needed page break in the table of contents..
% step 3
% insert  \addtocontents{toc}{\protect\newpage} before the first line i.e. before the line \chapter{RESULTS}.

%%%%%%%%%%%%%%%%%%%%%%%%%%%%
% \def\@makechapterhead#1{%   
% IN ORDER TO MAKE spacing changes in the title page got to the section in the isuthesis.cls file
% that starts with \long\def\maketitle{\begin{titlepage} and you can use options like
% singlespace (less spacing)
%singlespacing (comparitively more spacing almost like 2 spacing)
% onehalfspacing
%doublespacing (this is more spacing than the singlespacing above )
% more definitions on spacing can be found by going through the class file

% use \caption{} for all captions of figures and tables, where the captions are not too long.

% Use \caption[]{} with the square brackets for short caption of figure or table that goes into the list of tables and list of figures, and the curly brackets can have long captions which go with the figure/ table.
