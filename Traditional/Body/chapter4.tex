% Chapter 4 from the standard thesis template
%   that contains an adv. example table and figure.
%\addtocontents{toc}{\protect\newpage}
\chapter{RESULTS}

This is the opening paragraph to my thesis which
explains in general terms the concepts and hypothesis
which will be used in my thesis.

With more general information given here than really
necessary.

\section{Introduction}

Here initial concepts and conditions are explained and
several hypothesis are mentioned in brief.

Of course, data on this as seen in \autoref{data}
is few and far between.

\begin{table}[h!tb] \centering
  \caption{Moon Data}
  \label{data}
  % Use: \begin{tabular{|lcc|} to put table in a box
  %use \tagpdfsetup{table/tagging=presentation} if and only if the table
  %is for alignment/presentation purposes and is not a real table
  %
  %\tagpdfsetup{table/header-rows={1,2}} would have rows 1 and 2 be header rows
  \tagpdfsetup{table/header-rows={1}}
  %Use \tagpdfsetup{table/header-columns={}} for header columns instead
  %Put \tagpdfsetup{table/multirow={⟨number of rows⟩}} in cells spanning multiple rows
  %Note, multicolumn works automatically (no extra commands necessary)
  \begin{tabular}{lcc} \hline
    \textbf{Element} & \textbf{Control} & \textbf{Experimental} \\ \hline
    Moon Rings       & 1.23             & 3.38                  \\
    Moon Tides       & 2.26             & 3.12                  \\
    Moon Walk        & 3.33             & 9.29                  \\ \hline
  \end{tabular}
\end{table}

\subsection{Hypothesis}

Here one particular hypothesis is explained in depth
and is examined in the light of current literature.

Or graphically as seen in \autoref{mgraph}
it is certain that my hypothesis is true.

\begin{figure}[h!tb] \centering

  \includegraphics[alt={Picture of Durham you know}]{Images/dc5}

  \caption{Durham Centre}
  \label{mgraph}
\end{figure}

\subsubsection{Parts of the hypothesis}

Here one particular part of the hypothesis that is
currently being explained is examined and particular
elements of that part are given careful scrutiny.

% Below \subsubsection
% Sectional commands: \paragraph and \subparagraph may also be used

\subsection{Second Hypothesis}

Here one particular hypothesis is explained in depth
and is examined in the light of current literature.

\subsubsection{Parts of the second hypothesis}

Here one particular part of the hypothesis that is
currently being explained is examined and particular
elements of that part are given careful scrutiny.

%similar alt text works with e.g. \begin{tikzcd} or \begin{picture}
%a tikz picture that will not move
\begin{center}
  \captionsetup{type=figure}%if you don't using floating figures, you must tell latex that it is a figure
  \begin{tikzpicture}[alt={tikz graphic that works natively on the June version}]
    \draw[gray, thick] (-1,2) -- (2,-4);
    \draw[gray, thick] (-1,-1) -- (2,2);
    \filldraw[black] (0,0) circle (2pt) node[anchor=west]{A Tikz Picture that stays put! };
  \end{tikzpicture}
  \caption{A caption for a tikz picture}
\end{center}
Some more text here you see.
%a floating tikz instead
\begin{figure}[tb]
  \begin{tikzpicture}[alt={tikz graphic that works natively on the June version that floats}]
    \draw[gray, thick] (-1,2) -- (2,-4);
    \draw[gray, thick] (-1,-1) -- (2,2);
    \filldraw[black] (0,0) circle (2pt) node[anchor=west]{A Tikz Picture that floats! };
  \end{tikzpicture}
  \caption{A caption for a tikz picture}
\end{figure}

\section{Criteria Review}

Here certain criteria are explained thus eventually
leading to a foregone conclusion.
