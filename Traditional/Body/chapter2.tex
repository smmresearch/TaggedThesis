% Chapter 2 of the Thesis Template File
%   which includes bibliographic references.
\chapter{Review of Literature}
% Chapter titles are to be all caps and are automatically formatted so.
% If you have an exception and need lower-case letters outside of math (which are automatically ignored),
% you can use \NoCaseChange{....} around what should be protected

This is the opening paragraph to my thesis which
explains in general terms the concepts and hypothesis
which will be used in my thesis.

With more general information given here than really
necessary.

\section{Introduction}

Here initial concepts and conditions are explained and
several hypothesis are mentioned in brief.

did the initial work in this area. But in Struss' work \autocite{buiEveryGeneratingPolytope2023}
the definitive model is seen.

\subsection{Hypothesis}

Here one particular hypothesis is explained in depth
and is examined in the light of current literature.

\subsubsection{Parts of the hypothesis}

Here one particular part of the hypothesis that is
currently being explained is examined and particular
elements of that part are given careful scrutiny.

% Below \subsubsection
% Sectional commands: \paragraph and \subparagraph may also be used

\subsection{Second Hypothesis}

\paragraph{Heading} Here one particular hypothesis is explained in depth
and is examined in the light of current literature. \subparagraph{Even smaller heading} Another sentence.

\subsubsection{Parts of the second hypothesis}

Here one particular part of the hypothesis that is
currently being explained is examined and particular
elements of that part are given careful scrutiny.

\section{Criteria Review}

Here certain criteria are explained thus eventually
leading to a foregone conclusion.
\section{Continuing Tables}
Note, tables with cells spanning multiple columns work automatically, but cells spanning multiple rows require extra tagging.
\begin{table}[ht]
  \caption{This is a two-part table that also has cells spanning multiple rows.}
  %use \tagpdfsetup{table/tagging=presentation} if and only if the table
  %is for alignment/presentation purposes and is not a real table
  %
  %\tagpdfsetup{table/header-rows={1,2}} would have rows 1 and 2 be header rows
  \tagpdfsetup{table/header-rows={1}}
  %Use \tagpdfsetup{table/header-columns={}} for header columns instead
  %Put \tagpdfsetup{table/multirow={⟨number of rows⟩}} in cells spanning multiple rows
  %Note, multicolumn works automatically (no extra commands necessary)
  \begin{tabular}{rrrlrrrr}
    k      & q          & p+ & p- & s1                                                         & s2                                                         & s3                                                         & RHS       \\ \hline
    2      & 2          & 2  & 1  & 1                                                          & 0                                                          & 0                                                          & 1         \\
    -T     & 0          & 1  & 1  & 0                                                          & 1                                                          & 0                                                          & 0         \\
    T      & -1         & 0  & 1  & 0                                                          & 0                                                          & 1                                                          & 0         \\
    -1     & 1          & -1 & 1  &                                                            &                                                            &                                                            &           \\ \hline
    2(T+1) & 2          & 0  & 1  & 1                                                          & -2                                                         & 0                                                          & 1         \\ \hline
    -T     & 0          & 1  & 1  & 0                                                          & 1                                                          & 0                                                          & 0         \\
    T      & -1         & 0  & 1  & 0                                                          & 0                                                          & 1                                                          & 0         \\
    -(T+1) & 1          & 0  & 1  & 0                                                          & 1                                                          & 0                                                          &           \\ \hline
    0      & 2+2(T+1)/T & 0  & 1  & 1                                                          & -2                                                         & -2(T+1)/T                                                  & 1         \\ \hline
    0      & -1         & 1  & 1  & 0                                                          & 1                                                          & 1                                                          & 0         \\
    1      & -1/T       & 0  & 1  & 0                                                          & 0                                                          & 1/T                                                        & 0         \\
    0      & 1-(T+1)/T  & 0  & 1  & 0                                                          & 1                                                          & (T+1)/T                                                    &           \\ \hline
    0      & 2(2T+1)/T  & 0  & 1  & 1                                                          & -2                                                         & -2(T+1)/T                                                  & 1         \\ \hline
    0      & -1         & 1  & 1  & 0                                                          & 1                                                          & 1                                                          & 0         \\
    1      & -1/T       & 0  & 1  & 0                                                          & 0                                                          & 1/T                                                        & 0         \\
    0      & -1/T       & 0  & 1  & 0                                                          & 1                                                          & (T+1)/T                                                    &           \\ \hline
    0      & 1          & 0  & 1  & T/2(2T+1)                                                  & -T/(2T+1)                                                  & -1                                                         & T/2(2T+1) \\ \hline
    0      & 0          & 1  & 1  & T/2(2T+1)                                                  & 1-T/(2T+1)                                                 & 0                                                          & T/2(2T+1) \\
    1      & 0          & 0  & 1  & 1/2(2T+1)                                                  & -1/(2T+1)                                                  & 0                                                          & 1/2(2T+1) \\ \hline
    0      & 0          & 0  & 1  & 1/2(2T+1)                                                  & 1-1/(2T+1)                                                 & -1+(T+1)/TT                                                &           \\ \hline
    0      & 0          & 0  & 1  & 1/2(2T+1)                                                  & 1-1/(2T+1)                                                 & -1+(T+1)/TT                                                &           \\ \hline
    0      & 0          & 0  & 0  & \tagpdfsetup{table/multirow={8}}\multirow{8}{*}{1/2(2T+1)} & \tagpdfsetup{table/multirow={8}}\multirow{8}{*}{1/2(2T+1)} & \tagpdfsetup{table/multirow={8}}\multirow{8}{*}{1/2(2T+1)} &           \\ \cline{1-4}
    0      & 0          & 0  & 0  &                                                            &                                                            &                                                            &           \\ \cline{1-4}
    0      & 0          & 0  & 0  &                                                            &                                                            &                                                            &           \\ \cline{1-4}
    0      & 0          & 0  & 0  &                                                            &                                                            &                                                            &           \\ \cline{1-4}
    0      & 0          & 0  & 0  &                                                            &                                                            &                                                            &           \\ \cline{1-4}
    0      & 0          & 0  & 0  &                                                            &                                                            &                                                            &           \\ \cline{1-4}
    0      & 0          & 0  & 0  &                                                            &                                                            &                                                            &           \\ \cline{1-4}
    0      & 0          & 0  & 0  &                                                            &                                                            &                                                            &           \\ \hline
  \end{tabular}
\end{table}
%\tagpdfsetup{table/header-rows={1,2}} would have rows 1 and 2 be header rows
\tagpdfsetup{table/header-rows={1}}
%Use \tagpdfsetup{table/header-columns={}} for header columns instead
%Put \tagpdfsetup{table/multirow={⟨number of rows comma separated ⟩} in cells spanning multiple rows
\begin{table}[ht]
  %As a continued table, use \ContinuedFloat to keep the same #
  %And use the text "Continued" inside the caption
  \ContinuedFloat
  \caption{Continued}
  %\tagpdfsetup{table/header-rows={1,2}} would have rows 1 and 2 be header rows
  \tagpdfsetup{table/header-rows={1}}
  %Use \tagpdfsetup{table/header-columns={}} for header columns instead
  %Put \tagpdfsetup{table/multirow={⟨number of rows comma separated ⟩} in cells spanning multiple rows

  \begin{tabular}{rrrlrrrr}
    k      & q          & p+ & p- & s1        & s2         & s3          & RHS       \\ \hline
    2      & 2          & 2  & 1  & 1         & 0          & 0           & 1         \\
    -T     & 0          & 1  & 1  & 0         & 1          & 0           & 0         \\
    T      & -1         & 0  & 1  & 0         & 0          & 1           & 0         \\
    -1     & 1          & -1 & 1  &           &            &             &           \\ \hline
    2(T+1) & 2          & 0  & 1  & 1         & -2         & 0           & 1         \\ \hline
    -T     & 0          & 1  & 1  & 0         & 1          & 0           & 0         \\
    T      & -1         & 0  & 1  & 0         & 0          & 1           & 0         \\
    -(T+1) & 1          & 0  & 1  & 0         & 1          & 0           &           \\ \hline
    0      & 2+2(T+1)/T & 0  & 1  & 1         & -2         & -2(T+1)/T   & 1         \\ \hline
    0      & -1         & 1  & 1  & 0         & 1          & 1           & 0         \\
    1      & -1/T       & 0  & 1  & 0         & 0          & 1/T         & 0         \\
    0      & 1-(T+1)/T  & 0  & 1  & 0         & 1          & (T+1)/T     &           \\ \hline
    0      & 2(2T+1)/T  & 0  & 1  & 1         & -2         & -2(T+1)/T   & 1         \\ \hline
    0      & -1         & 1  & 1  & 0         & 1          & 1           & 0         \\
    1      & -1/T       & 0  & 1  & 0         & 0          & 1/T         & 0         \\
    0      & -1/T       & 0  & 1  & 0         & 1          & (T+1)/T     &           \\ \hline
    0      & 1          & 0  & 1  & T/2(2T+1) & -T/(2T+1)  & -1          & T/2(2T+1) \\ \hline
    0      & 0          & 1  & 1  & T/2(2T+1) & 1-T/(2T+1) & 0           & T/2(2T+1) \\
    1      & 0          & 0  & 1  & 1/2(2T+1) & -1/(2T+1)  & 0           & 1/2(2T+1) \\ \hline
    0      & 0          & 0  & 0  & 1/2(2T+1) & 1-1/(2T+1) & -1+(T+1)/TT &           \\ \hline
  \end{tabular}
\end{table}

