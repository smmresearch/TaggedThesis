%!LW recipe=latexmk (lualatex)
%The above command is for use with VSCode as an editor - can be deleted for others

%%%%%%% Instructions %%%%%%%%%%%%%%%%%%%

%Use lualatex to compile
%Therefore, do not use amssymb, input/output enc
% Get Error ".. \@ doesn't match definition.."? Try deleting all auxillary files.

% Did you know you can have the auxillary files in a different directory?
% If using latexmk (which you should), use command line argument of the form "-auxdir=%OUTDIR%/LatexAux" 
% You will also need the flag "-emulate-aux-dir" if using TexLive (MikTex natively supports auxillary directories)
% Makes deleting all of them a snap


%% Tagging is slow
%%
%% Things that can be commented or uncommented in this tex file
%% while drafting
%% (and should be un-done before submitting)
%% to speed-up compliation
%% 
%% 1. Use "\AddToHook{enddocument/end}{\tagpdfsetup{activate/tree=false}}"
    %Find this commented out above DocumentMetadata
    %This command disables a slow part of tagging
    %May want to disable periodically to ensure tagging is working
    %Comment out before submission
%% 2. Use "\tagpdfsetup{
%                  math/mathml/sources=,
%                  math/mathml/luamml/write=false
%                 }"
    %Find this command commented out below DocumentMetadata
    %This command disables math related accessiblity tagging (which is also slow)
    %Comment out before submission
% 3. Comment out "\usepackage[default]{fontsetup}" (unless using special characters - then you will need it)
    %Located in the middle of the preamble
    %This command sets up fonts to look nicer
    %Should be active (not a comment) before submission

%%%%%%%%%%%%%%%%%%%%%%%%%%%%%%%%%%%%%%%%%%%%%%%%%%%%%%%%%%%%%%%%%%%%%%%%%%%%%%%%%%%%%%%%%%%%%%%%%%%%%%%%%%%%%%%%%%%%%%%%%


%Command below disables slow parts of tagging,
%speeding up compiling during drafting stages
%When submitting, it should be commented out
%\AddToHook{enddocument/end}{\tagpdfsetup{activate/tree=false}}
%



\DocumentMetadata{
 lang=en,
 testphase={phase-III,math,table,title,firstaid},
 pdfversion=2.0,
 pdfstandard=ua-2,
 pdfstandard=a-4f,
 uncompress
}

%The command below disables the slow parts of tagging math, 
%speeding up compiling
%When submitting, it should be commented out
% \tagpdfsetup{
%  math/mathml/sources=,
%  math/mathml/luamml/write=false
% }


\documentclass[11pt]{isuthesistagged}



%Caption package & tagging fixes
\usepackage{caption}
\usepackage{subcaption}
%This work-around comes from: https://github.com/latex3/tagging-project/issues/720
\RemoveFromHook{begindocument}[latex-lab-testphase-float]
\makeatletter
\ExplSyntaxOn
\socket_new_plug:nnn{tagsupport/parbox/before}{caption}
  {   
   \tagpdfparaOn %restart para tagging
   \tl_if_empty:NTF\@current@float@struct
    {     
     \tag_struct_begin:n{tag=Caption,firstkid}
    }
    {
     \tag_struct_begin:n{tag=Caption,parent=\@current@float@struct,firstkid}
    } 
  }
\socket_new_plug:nnn{tagsupport/parbox/after}{caption}
  {
   \tag_struct_end:   
  }
\l@addto@macro\caption@beginex@hook{%
  \tagpdfparaOff %leavevmode in parbox should not start paragraph structure
  \AssignSocketPlug{tagsupport/parbox/before}{caption}
  \AssignSocketPlug{tagsupport/parbox/after}{caption}}
\ExplSyntaxOff
\AtBeginDocument{
	\renewcommand*\caption@anchor[1]{%
        \ifmeasuring@ \else
           \caption@raisedlink{\MakeLinkTarget*{#1}}%
        \fi}%  
}
\makeatother

%for tables
\usepackage{multirow}

%%Mathy packages
\usepackage{amsmath}
\usepackage{amsthm}
\usepackage{mathrsfs}
\usepackage{bm}
\usepackage{mathtools}
\usepackage{thmtools}
\usepackage{templatedShortcuts-private}
%Tiks? Externally generate as pdf and use includegraphics to add alt text
%until tikz implements alt text
%%


%This command sets-up some nice default font choices.
%The package fontspec can be used instead for further customization
%Commenting out font specifications while drafting or until otherwise strictly needed
%can speed up compilation
%It should be active (not a comment) when submitting for fonts to look their best

\usepackage[default]{fontsetup}

\usepackage{unicode-math}

%Change the font
% \setmainfont{texgyrepagella}[
% Extension       = .otf,
% UprightFont     = *-regular,
% ItalicFont      = *-italic,
% BoldFont        = *-bold,
% BoldItalicFont  = *-bolditalic 
% ]
% \setmathfont{texgyrepagella-math.otf}

\usepackage{graphicx}

\chaptertitle
% Old-style, thesis numbering down to subsubsection
\alternate
\usepackage{rotating}
% Bibliography without numbers or labels
%Natbib compatibility mode activated
%Different styles available - look at biblatex documentation
\usepackage[natbib=true,style=authoryear,style=apa]{biblatex}
\addbibresource{thesisAccessTest.bib}
\setlength{\bibitemsep}{13.2pt}

%Set the author
\newcommand{\theAuthor}{Alice Wonder}
%Set the tite 
\twoLineTitle{This is the title of the thesis
submitted to Iowa State University on the first line}{
Note that only the first letter of
the first word and proper names
are capitalized and this is the second line}

%%PDF properties automatically set
\author{\theAuthor}
\title{\titleWithLineBreak}
\usepackage[hypertexnames=false,linktocpage=true,pdfauthor={\theAuthor},pdftitle={\titleWithoutBreak},]{hyperref}
\hypersetup{colorlinks=true,linkcolor=blue,citecolor=blue,filecolor=blue,urlcolor=blue,bookmarksnumbered=true,pdfview=FitB,pdfencoding=auto}
\usepackage{bookmark}
% The following piece of code removes extra space on the top of each chapter
%  that is default of latex report class documents

\usepackage{etoolbox}
\makeatletter
\setlength{\@fptop}{0pt} %command added to ensure images always float on top of the page
\makeatother

%%%%%%%%%%%%%%%%%%%%%%%%%%%
\usepackage{float}
%%%%%%%%%%%%%%%%%%%%%%%%%%%%%%%%%%
\usepackage{xcolor}
\usepackage{graphicx}
\usepackage{geometry}
\geometry{letterpaper, left=1in, top=1in, right=1in, bottom=1in, includehead=true,headheight=14pt} 
\usepackage{pdflscape}
%%%%%%%%%%%%%%%%%%%%%%%%%%%%%%%%%%

\usepackage[intoc, english]{nomencl}
\doublespacing
\AtBeginEnvironment{table}{\singlespacing}% 
\AtBeginEnvironment{figure}{\singlespacing}% 

%%%Captioning Format
\DeclareCaptionListFormat{figureList}{Figure #1#2}
\DeclareCaptionListFormat{tableList}{Table #1#2}
\captionsetup[figure]{listformat=figureList}
\captionsetup[table]{listformat=tableList}
\captionsetup[ContinuedFloat]{list=no}


\usepackage[inline]{enumitem}


%The command below can be used to get rid of footnote rule(line)
%\renewcommand*\footnoterule{}

\begin{document}
\DeclareGraphicsExtensions{.jpg,.pdf,.mps,.png}
%\begin{singlespace}
\def\@makechapterheada{\vspace*{-2cm}\titlepage} % in order to reduce the space between margin and heading in titlepage
% Template Titlepage File
% Please choose appropriate options for Master's thesis, Doctoral dissertations, and creative components. Please read the comments to make an informed choice

\vspace*{-2cm}\titlepage
%%%%%%%%%%%%%%%%%%%

\degree{DOCTOR OF PHILOSOPHY}
\major{Mathematics}
%For co-majors use this command instead of major. Separate the co-majors with a semi-colon
%\comajors{Statistics; Computer Science}

\mprof{John Smith}
% If you have co-major professors, use \mprofs for the first, and \cmprofs for the second 
% instead of mprof
%\mprofs{ABC}
%\cmprofs{DEF}

\format{dissertation}%change to thesis or creative component as needed for a master's
\members{Jane Dee \\ Allen Wrench\\ Katniss Everdeen}
\disclaimertitlepage %sets-up title page to add disclaimer at end
\notice %sets-up title page to add the copyright notice at the bottom

%%%%%%
% "A <format> submitted to <the graduate faculty>..." can
% be changed to something other than graduate faculty with
% this command:
% \submit{the graduate faculty alternative}

\maketitle

\fancypagestyle{plain}{}


% % Left-justified setting for all sections including
% % dedication, nomenclature, acknowledgement, abstract and all chapters
% % Re-position the two lines below will change all the section
% % being compiled after those two lines
\raggedright
\parindent 0.25 in % set all paragraphs in the document to have indent
% %%%%%%%%%%%%%%%%%%%%%%%%%%%%%%%%%%%%%%%%%
% %% The line below adds the word "Page" over the page numbers in TOC, LOT, LOF
\addtocontents{toc}{~\hfill\tagstructbegin{tag=Lbl,stash,label=pageOfTOC}\tagmcbegin{tag=Lbl}\textbf{Page}\par\tagmcend\tagstructend}
\addtocontents{lot}{~\hfill\tagstructbegin{tag=Lbl,stash,label=pageOfLOT}\tagmcbegin{tag=Lbl}\textbf{Page}\par\tagmcend\tagstructend}
\addtocontents{lof}{~\hfill\tagstructbegin{tag=Lbl,stash,label=pageOfLOF}\tagmcbegin{tag=Lbl}\textbf{Page}\par\tagmcend\tagstructend}

% %%
% % Optional thesis dedication
%   Dedication is not usually listed in the table of contents.
%   However, if you do want it, add this command here (not in the dedication file)
%   \addToTOCWithoutChapter{DEDICATION}
\include{Preface/dedication}
% % Table of Contents, List of Tables and List of Figures

{
\pdfbookmark[0]{TABLE OF CONTENTS}{table}
\tableofcontents
\tagstructuse{pageOfTOC} %gets tagging of "Page" done
}

\addtocontents{toc}{\def\protect\@chapapp{}} \cleardoublepage \phantomsection
\pagebreak
% %%%%%%%%%%%%%%%%%%%%%%%%%%%%%%%%%%%%%%%%%
\MakeLinkTarget[specialchapter]{}%necessary for tagging (as of 2024)
\addcontentsline{toc}{chapter}{LIST OF TABLES}
\listoftables
\tagstructuse{pageOfLOT}%gets tagging of "Page" done
% %%%%%%%%%%%%%%%%%%%%%%%%%%%%%%%%%%%%%%%%%
\cleardoublepage \phantomsection
\MakeLinkTarget[specialchapter]{} %necessary for tagging (as of 2024)
\addcontentsline{toc}{chapter}{LIST OF FIGURES}
\listoffigures
\tagstructuse{pageOfLOF}%gets tagging of "Page" done
% %%%%%%%%%%%%%%%%%%%%%%%%%%%%%%%%%%%%%%%%%


% %Optional Nomenclature
\cleardoublepage \phantomsection
\MakeLinkTarget[specialchapter]{}
\makenomenclature
\renewcommand{\nomname}{NOMENCLATURE}
%\specialchapt{NOMENCLATURE}

%\mbox{}
\renewcommand\nomgroup[1]{%
  \item[\bfseries
  \ifstrequal{#1}{A}{Physics Constants}{%
  \ifstrequal{#1}{B}{Number Sets}{%
  \ifstrequal{#1}{C}{Other Symbols}{}}}%
]}

\nomenclature[A, 02]{$c$}{Speed of light in a vacuum inertial system}
\nomenclature[A, 03]{$h$}{Plank Constant}
\nomenclature[A, 01]{$g$}{Gravitational Constant}
\nomenclature[B, 03]{$\mathbb{R}$}{Real Numbers}
\nomenclature[B, 02]{$\mathbb{C}$}{Complex Numbers}
\nomenclature[B, 01]{$\mathbb{H}$}{Octonions}
\nomenclature[C]{$V$}{Constant Volume}
\nomenclature[C]{$\rho$}{Friction Index}

\renewcommand{\nompreamble}{The nomenclature for your dissertation or thesis is optional. This list may be placed in
the following places: as the last preliminary page, before the Reference section, or as an Appendix. The heading is bold if other major headings are bold, and the list is in the same font size and style as text. Nomenclature should follow a two-column format with the term in the left
column and its definition or description within the right column.}

\printnomenclature

% The following link has more tweaks, tips and tricks on how to setup nomenclatures: https://www.overleaf.com/learn/latex/Nomenclatures

%Adds Chapter in front of chapter on TOC
\addtocontents{toc}{\def\protect\@chapapp{CHAPTER\ }}


%Optional Acknowledgements
\cleardoublepage \phantomsection
\include{Preface/acknowl}
%Optional thesis abstract
\cleardoublepage \phantomsection
\include{Preface/abstract}
\cleardoublepage \phantomsection

\newpage
\pagenumbering{arabic}
\pagestyle{fancy}
% Chapter 1 of the Thesis Template File
\chapter{OVERVIEW}

This is the opening paragraph to my thesis which
explains in general terms the concepts and hypothesis
which will be used in my thesis.

With more general information given here than really
necessary.

\section{Introduction} \label{introSection}

Here initial concepts and conditions are explained and several hypothesis are mentioned in brief.

\subsection{Hypothesis}

Here one particular hypothesis is explained in depth and is examined in the light of current literature.

\subsubsection{Parts of the hypothesis}

Here one particular part of the hypothesis that is 
currently being explained is examined and particular
elements of that part are given careful scrutiny.

% Below \subsubsection
% Sectional commands: \paragraph and \subparagraph may also be used

\subsection{Second Hypothesis}

Here one particular hypothesis is explained in depth
and is examined in the light of current literature.

\subsubsection{Parts of the second hypothesis}

Here one particular part of the hypothesis that is 
currently being explained is examined and particular
elements of that part are given careful scrutiny.

\section{Criteria Review}

Here certain criteria are explained thus eventually
leading to a foregone conclusion.



\cite{correaValadierlikeFormulasSupremum},\cite{kleeHellyTheoremIts1963}
%\chapterbib
%\bibliographystyle{apa}
%\bibliography{Reference/mybib}


%\renewcommand{\bibname}{\centerline{BIBLIOGRAPHY}}
%\bibliographystyle{apa}
%\newpage
%\phantomsection
%\addtocontents{toc}{\def\protect\@chapapp{}}
%\addcontentsline{toc}{chapter}{BIBLIOGRAPHY}
%\addtocontents{toc}{\def\protect\@chapapp{CHAPTER\ }}
%\bibliography{mybib}
% Chapter 2 of the Thesis Template File
%   which includes bibliographic references.
\chapter{Review of Literature}
% Chapter titles are to be all caps and are automatically formatted so.
% If you have an exception and need lower-case letters outside of math (which are automatically ignored),
% you can use \NoCaseChange{....} around what should be protected

This is the opening paragraph to my thesis which
explains in general terms the concepts and hypothesis
which will be used in my thesis.

With more general information given here than really
necessary.

\section{Introduction}

Here initial concepts and conditions are explained and
several hypothesis are mentioned in brief.

did the initial work in this area. But in Struss' work \autocite{buiEveryGeneratingPolytope2023}
the definitive model is seen.

\subsection{Hypothesis}

Here one particular hypothesis is explained in depth
and is examined in the light of current literature.

\subsubsection{Parts of the hypothesis}

Here one particular part of the hypothesis that is
currently being explained is examined and particular
elements of that part are given careful scrutiny.

% Below \subsubsection
% Sectional commands: \paragraph and \subparagraph may also be used

\subsection{Second Hypothesis}

\paragraph{Heading} Here one particular hypothesis is explained in depth
and is examined in the light of current literature. \subparagraph{Even smaller heading} Another sentence.

\subsubsection{Parts of the second hypothesis}

Here one particular part of the hypothesis that is
currently being explained is examined and particular
elements of that part are given careful scrutiny.

\section{Criteria Review}

Here certain criteria are explained thus eventually
leading to a foregone conclusion.
\section{Continuing Tables}
Note, tables with cells spanning multiple columns work automatically, but cells spanning multiple rows require extra tagging.
\begin{table}[ht]
  \caption{This is a two-part table that also has cells spanning multiple rows.}
  %use \tagpdfsetup{table/tagging=presentation} if and only if the table
  %is for alignment/presentation purposes and is not a real table
  %
  %\tagpdfsetup{table/header-rows={1,2}} would have rows 1 and 2 be header rows
  \tagpdfsetup{table/header-rows={1}}
  %Use \tagpdfsetup{table/header-columns={}} for header columns instead
  %Put \tagpdfsetup{table/multirow={⟨number of rows⟩}} in cells spanning multiple rows
  %Note, multicolumn works automatically (no extra commands necessary)
  \begin{tabular}{rrrlrrrr}
    k      & q          & p+ & p- & s1                                                         & s2                                                         & s3                                                         & RHS       \\ \hline
    2      & 2          & 2  & 1  & 1                                                          & 0                                                          & 0                                                          & 1         \\
    -T     & 0          & 1  & 1  & 0                                                          & 1                                                          & 0                                                          & 0         \\
    T      & -1         & 0  & 1  & 0                                                          & 0                                                          & 1                                                          & 0         \\
    -1     & 1          & -1 & 1  &                                                            &                                                            &                                                            &           \\ \hline
    2(T+1) & 2          & 0  & 1  & 1                                                          & -2                                                         & 0                                                          & 1         \\ \hline
    -T     & 0          & 1  & 1  & 0                                                          & 1                                                          & 0                                                          & 0         \\
    T      & -1         & 0  & 1  & 0                                                          & 0                                                          & 1                                                          & 0         \\
    -(T+1) & 1          & 0  & 1  & 0                                                          & 1                                                          & 0                                                          &           \\ \hline
    0      & 2+2(T+1)/T & 0  & 1  & 1                                                          & -2                                                         & -2(T+1)/T                                                  & 1         \\ \hline
    0      & -1         & 1  & 1  & 0                                                          & 1                                                          & 1                                                          & 0         \\
    1      & -1/T       & 0  & 1  & 0                                                          & 0                                                          & 1/T                                                        & 0         \\
    0      & 1-(T+1)/T  & 0  & 1  & 0                                                          & 1                                                          & (T+1)/T                                                    &           \\ \hline
    0      & 2(2T+1)/T  & 0  & 1  & 1                                                          & -2                                                         & -2(T+1)/T                                                  & 1         \\ \hline
    0      & -1         & 1  & 1  & 0                                                          & 1                                                          & 1                                                          & 0         \\
    1      & -1/T       & 0  & 1  & 0                                                          & 0                                                          & 1/T                                                        & 0         \\
    0      & -1/T       & 0  & 1  & 0                                                          & 1                                                          & (T+1)/T                                                    &           \\ \hline
    0      & 1          & 0  & 1  & T/2(2T+1)                                                  & -T/(2T+1)                                                  & -1                                                         & T/2(2T+1) \\ \hline
    0      & 0          & 1  & 1  & T/2(2T+1)                                                  & 1-T/(2T+1)                                                 & 0                                                          & T/2(2T+1) \\
    1      & 0          & 0  & 1  & 1/2(2T+1)                                                  & -1/(2T+1)                                                  & 0                                                          & 1/2(2T+1) \\ \hline
    0      & 0          & 0  & 1  & 1/2(2T+1)                                                  & 1-1/(2T+1)                                                 & -1+(T+1)/TT                                                &           \\ \hline
    0      & 0          & 0  & 1  & 1/2(2T+1)                                                  & 1-1/(2T+1)                                                 & -1+(T+1)/TT                                                &           \\ \hline
    0      & 0          & 0  & 0  & \tagpdfsetup{table/multirow={8}}\multirow{8}{*}{1/2(2T+1)} & \tagpdfsetup{table/multirow={8}}\multirow{8}{*}{1/2(2T+1)} & \tagpdfsetup{table/multirow={8}}\multirow{8}{*}{1/2(2T+1)} &           \\ \cline{1-4}
    0      & 0          & 0  & 0  &                                                            &                                                            &                                                            &           \\ \cline{1-4}
    0      & 0          & 0  & 0  &                                                            &                                                            &                                                            &           \\ \cline{1-4}
    0      & 0          & 0  & 0  &                                                            &                                                            &                                                            &           \\ \cline{1-4}
    0      & 0          & 0  & 0  &                                                            &                                                            &                                                            &           \\ \cline{1-4}
    0      & 0          & 0  & 0  &                                                            &                                                            &                                                            &           \\ \cline{1-4}
    0      & 0          & 0  & 0  &                                                            &                                                            &                                                            &           \\ \cline{1-4}
    0      & 0          & 0  & 0  &                                                            &                                                            &                                                            &           \\ \hline
  \end{tabular}
\end{table}
%\tagpdfsetup{table/header-rows={1,2}} would have rows 1 and 2 be header rows
\tagpdfsetup{table/header-rows={1}}
%Use \tagpdfsetup{table/header-columns={}} for header columns instead
%Put \tagpdfsetup{table/multirow={⟨number of rows comma separated ⟩} in cells spanning multiple rows
\begin{table}[ht]
  %As a continued table, use \ContinuedFloat to keep the same #
  %And use the text "Continued" inside the caption
  \ContinuedFloat
  \caption{Continued}
  %\tagpdfsetup{table/header-rows={1,2}} would have rows 1 and 2 be header rows
  \tagpdfsetup{table/header-rows={1}}
  %Use \tagpdfsetup{table/header-columns={}} for header columns instead
  %Put \tagpdfsetup{table/multirow={⟨number of rows comma separated ⟩} in cells spanning multiple rows

  \begin{tabular}{rrrlrrrr}
    k      & q          & p+ & p- & s1        & s2         & s3          & RHS       \\ \hline
    2      & 2          & 2  & 1  & 1         & 0          & 0           & 1         \\
    -T     & 0          & 1  & 1  & 0         & 1          & 0           & 0         \\
    T      & -1         & 0  & 1  & 0         & 0          & 1           & 0         \\
    -1     & 1          & -1 & 1  &           &            &             &           \\ \hline
    2(T+1) & 2          & 0  & 1  & 1         & -2         & 0           & 1         \\ \hline
    -T     & 0          & 1  & 1  & 0         & 1          & 0           & 0         \\
    T      & -1         & 0  & 1  & 0         & 0          & 1           & 0         \\
    -(T+1) & 1          & 0  & 1  & 0         & 1          & 0           &           \\ \hline
    0      & 2+2(T+1)/T & 0  & 1  & 1         & -2         & -2(T+1)/T   & 1         \\ \hline
    0      & -1         & 1  & 1  & 0         & 1          & 1           & 0         \\
    1      & -1/T       & 0  & 1  & 0         & 0          & 1/T         & 0         \\
    0      & 1-(T+1)/T  & 0  & 1  & 0         & 1          & (T+1)/T     &           \\ \hline
    0      & 2(2T+1)/T  & 0  & 1  & 1         & -2         & -2(T+1)/T   & 1         \\ \hline
    0      & -1         & 1  & 1  & 0         & 1          & 1           & 0         \\
    1      & -1/T       & 0  & 1  & 0         & 0          & 1/T         & 0         \\
    0      & -1/T       & 0  & 1  & 0         & 1          & (T+1)/T     &           \\ \hline
    0      & 1          & 0  & 1  & T/2(2T+1) & -T/(2T+1)  & -1          & T/2(2T+1) \\ \hline
    0      & 0          & 1  & 1  & T/2(2T+1) & 1-T/(2T+1) & 0           & T/2(2T+1) \\
    1      & 0          & 0  & 1  & 1/2(2T+1) & -1/(2T+1)  & 0           & 1/2(2T+1) \\ \hline
    0      & 0          & 0  & 0  & 1/2(2T+1) & 1-1/(2T+1) & -1+(T+1)/TT &           \\ \hline
  \end{tabular}
\end{table}


% Chapter 3 from the thesis template file
%   that contains an example table and figure.
\chapter{Methods and Procedures}
% Chapter titles are to be all caps and are automatically formatted so.
% If you have an exception and need lower-case letters outside of math (which are automatically ignored),
% you can use \NoCaseChange{....} around what should be protected

This is the opening paragraph to my thesis which
explains in general terms the concepts and hypothesis
which will be used in my thesis.

With more general information given here than really
necessary.

\section{Introduction}

Here initial concepts and conditions are explained and
several hypothesis are mentioned in brief.

As can be seen in \autoref{nothing} it is truly
obvious what I am saying is true.

\begin{table}[h!tb] \centering

    \caption[Short caption for List of Figures/ Tables]{This table shows a standard empty table \autocite{kleeHellyTheoremIts1963}. Remove the square bracketed information to get longer captions in the LOT/ LOF }
    \label{nothing}

    \vspace{ 2 in}
\end{table}

\subsection{Hypothesis}

Here one particular hypothesis is explained in depth
and is examined in the light of current literature.

This can also be seen in \autoref{moon} that the
rest is obvious.

\begin{figure}[h!tb] \centering

    \includegraphics[alt={Alt text describing}]{Images/dc5.jpg}
    \caption[Short caption for List of Figures/ Tables]{This table shows a standard empty figure. Remove the square bracketed information to get longer captions in the LOT/ LOF}
    \label{moon}
\end{figure}

\subsubsection{Parts of the hypothesis}

Here one particular part of the hypothesis that is
currently being explained is examined and particular
elements of that part are given careful scrutiny.

% Below \subsubsection
% Sectional commands: \paragraph and \subparagraph may also be used

\subsection{Second Hypothesis}

Here one particular hypothesis is explained in depth
and is examined in the light of current literature.

\subsubsection{Parts of the second hypothesis}

Here one particular part of the hypothesis that is
currently being explained is examined and particular
elements of that part are given careful scrutiny.

%\addtocontents{toc}{\protect\newpage} % Adds \newpage in "\tableofcontents"
\section{Criteria Review}

Here certain criteria are explained thus eventually
leading to a foregone conclusion as can be seen in
\autoref{nevermore}.
\begin{table}[h!tb] \centering
    \captionsetup{width=2in}
    \caption{This table shows a standard table with a limited caption width}
    %
    %use \tagpdfsetup{table/tagging=presentation} if and only if the table
    %is for alignment/presentation purposes and is not a real table
    %
    %\tagpdfsetup{table/header-rows={1,2}} would have rows 1 and 2 be header rows
    \tagpdfsetup{table/header-rows={1}}
    %Use \tagpdfsetup{table/header-columns={}} for header columns instead
    %Put \tagpdfsetup{table/multirow={⟨number of rows⟩}} in cells spanning multiple rows
    %Note, multicolumn works automatically (no extra commands necessary)

    \begin{tabular}{c|c|c}
        Header & head & head \\
        leg    & leg  & leg  \\
        leg    & leg  & leg  \\
        leg    & leg  & leg  \\
    \end{tabular}
    \label{nevermore}
    \vspace{ 2 in}
\end{table}


\part{Let's have a part page}
% Chapter 4 from the standard thesis template
%   that contains an adv. example table and figure.
 %\addtocontents{toc}{\protect\newpage}
\chapter{RESULTS}

This is the opening paragraph to my thesis which
explains in general terms the concepts and hypothesis
which will be used in my thesis.

With more general information given here than really
necessary.

\section{Introduction}

Here initial concepts and conditions are explained and
several hypothesis are mentioned in brief.

Of course, data on this as seen in \autoref{data}
is few and far between.

\begin{table}[h!tb] \centering
\caption{Moon Data}
\label{data}
% Use: \begin{tabular{|lcc|} to put table in a box
%\tagpdfsetup{table/header-rows={1,2}} would have rows 1 and 2 be header rows
\tagpdfsetup{table/header-rows={1}}
%Use \tagpdfsetup{table/header-columns={}} for header columns instead
%Put \tagpdfsetup{table/multirow={⟨number of rows comma separated ⟩} in cells spanning multiple rows
\begin{tabular}{lcc} \hline
\textbf{Element} & \textbf{Control} & \textbf{Experimental} \\ \hline
Moon Rings & 1.23 & 3.38 \\
Moon Tides & 2.26 & 3.12 \\
Moon Walk & 3.33 & 9.29 \\ \hline
\end{tabular}
\end{table}


\subsection{Hypothesis}

Here one particular hypothesis is explained in depth
and is examined in the light of current literature.

Or graphically as seen in \autoref{mgraph}
it is certain that my hypothesis is true.

\begin{figure}[h!tb] \centering

\includegraphics[alt={Picture of Durham you know}]{Images/dc5}

\caption{Durham Centre}
\label{mgraph}
\end{figure}

\subsubsection{Parts of the hypothesis}

Here one particular part of the hypothesis that is 
currently being explained is examined and particular
elements of that part are given careful scrutiny.

% Below \subsubsection
% Sectional commands: \paragraph and \subparagraph may also be used

\subsection{Second Hypothesis}

Here one particular hypothesis is explained in depth
and is examined in the light of current literature.

\subsubsection{Parts of the second hypothesis}

Here one particular part of the hypothesis that is 
currently being explained is examined and particular
elements of that part are given careful scrutiny.

\section{Criteria Review}

Here certain criteria are explained thus eventually
leading to a foregone conclusion.

% Chapter 5 from the standard thesis template
%   with a full page figure and a sideways table.
\chapter{SUMMARY AND DISCUSSION}

This is the opening paragraph to my thesis which
explains in general terms the concepts and hypothesis
which will be used in my thesis.

With more general information given here than really
necessary.

\section{Introduction}

Here initial concepts and conditions are explained and
several hypothesis are mentioned in brief.

Or graphically as seen in \autoref{mgraph2}
it is certain that my hypothesis is true.

%\begin{figure}[p!] \centering

\begin{figure}[h!tb] \centering
    \includegraphics[alt={This is alt text}]{Images/dc5}

    \caption{Durham Centre---  Another View}
    \label{mgraph2}
\end{figure}

\subsection{Hypothesis}

Here one particular hypothesis is explained in depth
and is examined in the light of current literature.

As can be seen in \autoref{nothingelse} it is
truly obvious what I am saying is true.

%\addtocontents{lot}{\protect\newpage}
\begin{landscape}
    \hfill
    \vfill
    %h! necessary for it to be centered
    \begin{table}[h!] \centering
        \caption{This table shows almost nothing but is a
            sideways table and takes up a whole page by itself}
        \label{nothingelse}
        % Use: \begin{tabular{|lcc|} to put table in a box
        %use \tagpdfsetup{table/tagging=presentation} if and only if the table
        %is for alignment/presentation purposes and is not a real table
        %
        %\tagpdfsetup{table/header-rows={1,2}} would have rows 1 and 2 be header rows
        \tagpdfsetup{table/header-rows={1}}
        %Use \tagpdfsetup{table/header-columns={}} for header columns instead
        %Put \tagpdfsetup{table/multirow={⟨number of rows⟩} in cells spanning multiple rows
        \begin{tabular}{lcc} \hline
            \textbf{Element} & \textbf{Control} & \textbf{Experimental} \\ \hline
            Moon Rings       & 1.23             & 3.38                  \\
            Moon Tides       & 2.26             & 3.12                  \\
            Moon Walk        & 3.33             & 9.29                  \\ \hline
        \end{tabular}
    \end{table}
    \hfill
    \vfill
\end{landscape}

\subsubsection{Parts of the hypothesis}

Here one particular part of the hypothesis that is
currently being explained is examined and particular
elements of that part are given careful scrutiny.

% Below \subsubsection
% Sectional commands: \paragraph and \subparagraph may also be used

%\addtocontents{toc}{\protect\newpage} %% Remove this if needed, this lines forces the lines of the TOC starting with the below sub-heading "Critical Review" to go to the next page. Remove this formatting line as it will be required only if you want to force a table of contents entry to the next page along with the other subsequent entries.

\subsection{Second Hypothesis}

Here one particular hypothesis is explained in depth
and is examined in the light of current literature.

\subsubsection{Parts of the second hypothesis}

Here one particular part of the hypothesis that is
currently being explained is examined and particular
elements of that part are given careful scrutiny.

\section{Criteria Review}

Here certain criteria are explained thus eventually
leading to a foregone conclusion.

\section{Results And Discussion}

Here the results can be inserted


%Pulls the appendices and bibliography out of any Parts
\bookmarksetup{startatroot}
\chapter*{BIBLIOGRAPHY}
\addToTOCWithoutChapter{BIBLIOGRAPHY}
\printbibliography[heading=none]
% Appendix1 file from standard thesis template

% Use these lines for multiple appendices

\appendixtitle
\appendix

%%%%%%%%%%%

%Use this line instead for a single appendix
% \singleappendixtitle
%%%%%%%%%%

\chapter{ADDITIONAL MATERIAL}
\finishAppendixSetup%must come immediately after the first appendix chapter command

This is now the same as any other chapter except that
all sectioning levels below the chapter level must begin
with the *-form of a sectioning command.

\section*{More stuff}
\begin{equation}
	x^2+z=yz
\end{equation}
\begin{figure}
	\caption{Something something.}
\end{figure}
\begin{table}[h!tb] \centering
	\caption[This table shows a standard non-empty table. Please check the code caption for extended instructions]{This table shows a standard empty table. In case of long captions, we want to use the long caption as the description to the table and image but not use it in the table of contents and list of figures/ tables. In order to do this, there are two captions which have been provided, remove the first square bracket options if there is only one small caption. You can use citations like this to}
	\begin{tabular}{ll}
		Bach      & Cello Suite Number 1  \\
		Beethoven & Cello Sonata Number 3 \\
		Brahms    & Cello Sonata Number 1
	\end{tabular}
	\label{nothingagain}

	\vspace{ 2 in}
\end{table}
Supplemental material.
\begin{figure}[b]
	\begin{subfigure}[c]{0.495\textwidth}
		\centering\includegraphics[alt={sample image},width=0.99\textwidth]{example-image-c}%
		\subcaption{\label{fig:2a}}
	\end{subfigure}
	\begin{subfigure}[c]{0.495\textwidth}
		\centering{\includegraphics[alt={sample image},width=0.99\textwidth]{example-image-c}}%
		\subcaption{\label{fig:2b}}%
	\end{subfigure}%
	\caption{A figure with two subfigures: (a) first subfigure; (b) second subfigure.\label{fig:2}}
\end{figure}

\begin{algorithm}
	\caption{Score Algorithm}
	\begin{algorithmic}[1]
		\State {\textbf{Input: }{$s$ is a sensor }}
		\Statex
		\For{$j\in \{1,2,\ldots,15\}$}
		\State Randomly choose $5$ days
		\For{$x\in \{1,2,\ldots,1000\}$}
		\parstate{Set $a$ to be something in this very long state that will have to be wrapped quite possibly around and around and around}
		\EndFor
		\EndFor
	\end{algorithmic}
\end{algorithm}

 
% % Instruction for single appendix check instruction in Appendix/appendix1.tex on top of the file
% An example second appendix from the example thesis thesis.tex.
\chapter{STATISTICAL RESULTS}

This is now the same as any other chapter except that
all sectioning levels below the chapter level must begin
with the *-form of a sectioning command.

\section*{Supplemental Statistics}

More stuff.
\begin{equation}
    \label{greatAppEq}
    z^2-a=2y
\end{equation}
You see here I reference \autoref{greatAppEq}
\begin{theorem}
    Appendices are the best part.
    \label{appThm}
\end{theorem}
Now, I reference \autoref{appThm}.

\end{document}

% IMPORTANT NOTES
% TABLE OF CONTENTS :
% TOPIC 1:  If you need a page break follow the steps below
% step1
% check before which chapter in the table of contents you want a page break
% step 2
% go the folder "body". There open the chapter tex file that you noted needed page break in the table of contents..
% step 3
% insert  \addtocontents{toc}{\protect\newpage} before the first line i.e. before the line \chapter{RESULTS}.

%%%%%%%%%%%%%%%%%%%%%%%%%%%%
% \def\@makechapterhead#1{%   
% IN ORDER TO MAKE spacing changes in the title page got to the section in the isuthesis.cls file
% that starts with \long\def\maketitle{\begin{titlepage} and you can use options like
% singlespace (less spacing)
%singlespacing (comparitively more spacing almost like 2 spacing)
% onehalfspacing
%doublespacing (this is more spacing than the singlespacing above )
% more definitions on spacing can be found by going through the class file


% use \caption{} for all captions of figures and tables, where the captions are not too long.

% Use \caption[]{} with the square brackets for short caption of figure or table that goes into the list of tables and list of figures, and the curly brackets can have long captions which go with the figure/ table.

