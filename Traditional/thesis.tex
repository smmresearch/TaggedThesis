%!LW recipe=latexmk (lualatex)
%The above command is for use with VSCode as an editor - can be deleted for others

%%%%%%% Instructions %%%%%%%%%%%%%%%%%%%

%Use lualatex to compile with --interaction-mode=nonstopmode
%The commands I run are of the form (in the folder containing thesis.tex): 

%lualatex-dev --aux-directory=./LatexAux --synctex=1 --interaction=nonstopmode --c-style-errors .\thesis.tex
%biber thesis --output-directory=.\LatexAux  

%the -dev won't be necessary post June 2025 release

%biber's output-directory should match lualatex's aux-directory or both should be omitted. c-style-errors is optional.

% If using unicode with lualatex, do not use amssymb, input/output enc
% Don't take any warnings/error seriously until it persists over multiple compilations
% Get Error ".. \@ doesn't match definition.."? 
% First, make sure you have ran biber and re-ran lualatex
% Then, try deleting *.aux file of file you were messing with. If needed, escalate to deleting all auxillary files.

%%Using Tikz? Externally generate as pdf (with just normal presets/without tagging) and use includegraphics to add alt text


%%% While drafting, the class options no-mathml and/or no-tag-tree can speed up compiling.
% no-tag-tree cannot be used for final submission. mathml accessiblity tagging is not currently required by ISU

%%%%%%%%%%%%%%%%%%%%%%%%%%%%%%%%%%%%%%%%%%%%%%%%%%%%%%%%%%%%%%%%%%%%%%%%%%%%%%%%%%%%%%%%%%%%%%%%%%%%%%%%%%%%%%%%%%%%%%%%%

\DocumentMetadata{
 lang=en,
 testphase=latest,
 pdfversion=2.0,
 pdfstandard=ua-2,
 pdfstandard=a-4f,
 %uncompress %can be helpful for debugging
}


%Can pass additional options to packages loaded by the class
%with \PassOptionsToPackage{<options>}{<packagename>}
\PassOptionsToPackage{noend}{algpseudocode}


% Template file for a standard thesis
\documentclass[12pt,math-packages,algorithms-packages]{isuthesistagged}
%packages always loaded in cls file: xpatch, fancyhdr, titlesec, setspace, nowidow, caption
%           subcaption, geometry, graphicx
%biblatex and hyperref must also be loaded later in the preamble


%Additional class options:
% draft: disables most of tagging, loading of figures, and other "expensive" tasks. Makes compilation much faster. Ignore any warnings from tagpdf when this option is on.
%       Cannot be used for final submission. Automatically includes the option no-mathml and no-tag-tree.
% math-packages: loads amsmath,amsthm, mathtools, keytheorems, and associated tagging fixes
% algorithms-packages: loads algorithm, algpseudocode, fixes the numbering, also provides \parstate for long, wrapped states
% no-mathml: disables math tagging (not required by Iowa State)
% no-tag-tree: disables a slow part of tagging (must be re-enabled before submission)
% bib-no-break: by default, page breaks in the middle of bibliography entries are discouraged, but may happen for a particularly long entry (4+ lines).
%       With this option, a page break will never happen in the middle of an entry, no matter how long. This is "recommended" by not required
%       by Iowa State.




%For complex tables
\usepackage{multirow}
\usepackage{pdflscape}

%%Provides math presets: 
%    Sets up some common math presets like Theorem, Definition, etc. 
%    Easier to edit or exclude than the class file
\usepackage{templatedTheoremTypes}

%%


%%%%%%%Fonts

%Check package documentation for other font options
\usepackage[libertinus]{fontsetup}
%Instead of font setup, unicode-math with fontspec has more fine-tuned control
%\usepackage{unicode-math}

% \setmainfont{texgyrepagella}[
% Extension       = .otf,
% UprightFont     = *-regular,
% ItalicFont      = *-italic,
% BoldFont        = *-bold,
% BoldItalicFont  = *-bolditalic 
% ]
% \setmathfont{texgyrepagella-math.otf}

%With unicode, symbf should be used instead of bm
%This remaps \bm to prevent problems. There might be slightly different behavior.
\let\bm\symbf

%Instead of fontsetup and unicode-math, you
%can use lmodern for a classic (not-unicode) Computer Modern font
%However, without unicode, the accessibility of the math formulas
%is much worse (currently ISU has no requirements about
%the accessibility of math, so it is suitable for submission)
%In this case, you should also comment out \let\bm\symbf
%\usepackage{lmodern}\usepackage[T1]{fontenc} 
%%%%%%%%%%%%%

%Natbib compatibility mode activated
%Different styles available - look at biblatex documentation
\usepackage[natbib=true,style=authoryear]{biblatex}
\addbibresource{thesisAccessTest.bib}
\setlength{\bibitemsep}{13.2pt}

%Set the author
\newcommand{\theAuthor}{Alice Wonder}
%Set the tite 
\twoLineTitle{This is the title of a thesis
submitted to Iowa State University on the first line.}{
Second line, only the first letter of
the first word and names
are capitalized}

%%PDF properties automatically set
\author{\theAuthor}
\title{\titleWithLineBreak}
\usepackage[hypertexnames=false,linktocpage=true,pdfauthor={\theAuthor},pdftitle={\titleWithoutBreak},]{hyperref}
\hypersetup{colorlinks=true,linkcolor=blue,citecolor=blue,filecolor=blue,urlcolor=blue,bookmarksnumbered=true,pdfview=FitB,pdfencoding=auto}
\usepackage{bookmark}

%%%%%%%%%%%%%%%%%%%%%

%%%%%%%%%%%%%%%%%%%%%%%%%%%%%%%%%%
%Optional nomenclature package
\usepackage[notintoc,english]{nomencl}


%The command below can be used to get rid of footnote rule(line)
%\renewcommand*\footnoterule{}

%Tikz is a popular package for creating diagrams in Latex
%It is included to demonstrate how to write alt text
%It is *not* required if you are not otherwise using it
\usepackage{tikz}

\begin{document}
% Template Titlepage File
% Please choose appropriate options for Master's thesis, Doctoral dissertations, and creative components. Please read the comments to make an informed choice

\vspace*{-2cm}\titlepage
%%%%%%%%%%%%%%%%%%%

\degree{DOCTOR OF PHILOSOPHY}
\major{Mathematics}
%For co-majors use this command instead of major. Separate the co-majors with a semi-colon
%\comajors{Statistics; Computer Science}

\mprof{John Smith}
% If you have co-major professors, use \mprofs for the first, and \cmprofs for the second 
% instead of mprof
%\mprofs{ABC}
%\cmprofs{DEF}

\format{dissertation}%change to thesis or creative component as needed for a master's
\members{Jane Dee \\ Allen Wrench\\ Katniss Everdeen}
\disclaimertitlepage %sets-up title page to add disclaimer at end
\notice %sets-up title page to add the copyright notice at the bottom

%%%%%%
% "A <format> submitted to <the graduate faculty>..." can
% be changed to something other than graduate faculty with
% this command:
% \submit{the graduate faculty alternative}

\maketitle

\frontmattersetup
% %%
% % Optional thesis dedication
%   Dedication is not usually listed in the table of contents.
%   However, if you do want it, add this command here (not in the dedication file)
%   \addToTOCWithoutChapter{DEDICATION}
\include{Preface/dedication}

\tableofcontentsTagged
\cleardoublepage \phantomsection
\pagebreak

\listoftablesTagged %remove if no tables

\cleardoublepage \phantomsection
\listoffiguresTagged %remove if no figures

%%%% Optional nomenclature
%\cleardoublepage \phantomsection
%\nomenclatureTagged

%Adds Chapter in front of chapter on TOC
\addtocontents{toc}{\def\protect\@chapapp{CHAPTER\ }}

%Optional Acknowledgements
\cleardoublepage \phantomsection
\include{Preface/acknowl}
%Required thesis abstract
\cleardoublepage \phantomsection
\include{Preface/abstract}
\newpage
\pagenumbering{arabic}
\pagestyle{fancy}
% Chapter 1 of the Thesis Template File
\chapter{\MakeUppercase{Overview}}
% \MakeUppercase{} is conveninent, but if it causes undesired
%behavior (perhaps due to acronyms or math), you can just manually capitalize your chapter title
% \chaper{GENERAL INTRODUCTION}

This is the opening paragraph to my thesis which
explains in general terms the concepts and hypothesis
which will be used in my thesis.

With more general information given here than really
necessary.

\section{Introduction} \label{introSection}

Here initial concepts and conditions are explained and several hypothesis are mentioned in brief.

\subsection{Hypothesis}

Here one particular hypothesis is explained in depth and is examined in the light of current literature.

\subsubsection{Parts of the hypothesis}

Here one particular part of the hypothesis that is
currently being explained is examined and particular
elements of that part are given careful scrutiny.

% Below \subsubsection
% Sectional commands: \paragraph and \subparagraph may also be used

\subsection{Second Hypothesis}

Here one particular hypothesis is explained in depth
and is examined in the light of current literature.

\subsubsection{Parts of the second hypothesis}

Here one particular part of the hypothesis that is
currently being explained is examined and particular
elements of that part are given careful scrutiny.

\section{Criteria Review}

Here certain criteria are explained thus eventually
leading to a foregone conclusion.

\autocite{correaValadierlikeFormulasSupremum},\autocite{kleeHellyTheoremIts1963}

% Chapter 2 of the Thesis Template File
%   which includes bibliographic references.
\chapter{Review of Literature}
% Chapter titles are to be all caps and are automatically formatted so.
% If you have an exception and need lower-case letters outside of math (which are automatically ignored),
% you can use \NoCaseChange{....} around what should be protected

This is the opening paragraph to my thesis which
explains in general terms the concepts and hypothesis
which will be used in my thesis.

With more general information given here than really
necessary.

\section{Introduction}

Here initial concepts and conditions are explained and
several hypothesis are mentioned in brief.

did the initial work in this area. But in Struss' work \autocite{buiEveryGeneratingPolytope2023}
the definitive model is seen.

\subsection{Hypothesis}

Here one particular hypothesis is explained in depth
and is examined in the light of current literature.

\subsubsection{Parts of the hypothesis}

Here one particular part of the hypothesis that is
currently being explained is examined and particular
elements of that part are given careful scrutiny.

% Below \subsubsection
% Sectional commands: \paragraph and \subparagraph may also be used

\subsection{Second Hypothesis}

\paragraph{Heading} Here one particular hypothesis is explained in depth
and is examined in the light of current literature. \subparagraph{Even smaller heading} Another sentence.

\subsubsection{Parts of the second hypothesis}

Here one particular part of the hypothesis that is
currently being explained is examined and particular
elements of that part are given careful scrutiny.

\section{Criteria Review}

Here certain criteria are explained thus eventually
leading to a foregone conclusion.
\section{Continuing Tables}
Note, tables with cells spanning multiple columns work automatically, but cells spanning multiple rows require extra tagging.
\begin{table}[ht]
  \caption{This is a two-part table that also has cells spanning multiple rows.}
  %use \tagpdfsetup{table/tagging=presentation} if and only if the table
  %is for alignment/presentation purposes and is not a real table
  %
  %\tagpdfsetup{table/header-rows={1,2}} would have rows 1 and 2 be header rows
  \tagpdfsetup{table/header-rows={1}}
  %Use \tagpdfsetup{table/header-columns={}} for header columns instead
  %Put \tagpdfsetup{table/multirow={⟨number of rows⟩}} in cells spanning multiple rows
  %Note, multicolumn works automatically (no extra commands necessary)
  \begin{tabular}{rrrlrrrr}
    k      & q          & p+ & p- & s1                                                         & s2                                                         & s3                                                         & RHS       \\ \hline
    2      & 2          & 2  & 1  & 1                                                          & 0                                                          & 0                                                          & 1         \\
    -T     & 0          & 1  & 1  & 0                                                          & 1                                                          & 0                                                          & 0         \\
    T      & -1         & 0  & 1  & 0                                                          & 0                                                          & 1                                                          & 0         \\
    -1     & 1          & -1 & 1  &                                                            &                                                            &                                                            &           \\ \hline
    2(T+1) & 2          & 0  & 1  & 1                                                          & -2                                                         & 0                                                          & 1         \\ \hline
    -T     & 0          & 1  & 1  & 0                                                          & 1                                                          & 0                                                          & 0         \\
    T      & -1         & 0  & 1  & 0                                                          & 0                                                          & 1                                                          & 0         \\
    -(T+1) & 1          & 0  & 1  & 0                                                          & 1                                                          & 0                                                          &           \\ \hline
    0      & 2+2(T+1)/T & 0  & 1  & 1                                                          & -2                                                         & -2(T+1)/T                                                  & 1         \\ \hline
    0      & -1         & 1  & 1  & 0                                                          & 1                                                          & 1                                                          & 0         \\
    1      & -1/T       & 0  & 1  & 0                                                          & 0                                                          & 1/T                                                        & 0         \\
    0      & 1-(T+1)/T  & 0  & 1  & 0                                                          & 1                                                          & (T+1)/T                                                    &           \\ \hline
    0      & 2(2T+1)/T  & 0  & 1  & 1                                                          & -2                                                         & -2(T+1)/T                                                  & 1         \\ \hline
    0      & -1         & 1  & 1  & 0                                                          & 1                                                          & 1                                                          & 0         \\
    1      & -1/T       & 0  & 1  & 0                                                          & 0                                                          & 1/T                                                        & 0         \\
    0      & -1/T       & 0  & 1  & 0                                                          & 1                                                          & (T+1)/T                                                    &           \\ \hline
    0      & 1          & 0  & 1  & T/2(2T+1)                                                  & -T/(2T+1)                                                  & -1                                                         & T/2(2T+1) \\ \hline
    0      & 0          & 1  & 1  & T/2(2T+1)                                                  & 1-T/(2T+1)                                                 & 0                                                          & T/2(2T+1) \\
    1      & 0          & 0  & 1  & 1/2(2T+1)                                                  & -1/(2T+1)                                                  & 0                                                          & 1/2(2T+1) \\ \hline
    0      & 0          & 0  & 1  & 1/2(2T+1)                                                  & 1-1/(2T+1)                                                 & -1+(T+1)/TT                                                &           \\ \hline
    0      & 0          & 0  & 1  & 1/2(2T+1)                                                  & 1-1/(2T+1)                                                 & -1+(T+1)/TT                                                &           \\ \hline
    0      & 0          & 0  & 0  & \tagpdfsetup{table/multirow={8}}\multirow{8}{*}{1/2(2T+1)} & \tagpdfsetup{table/multirow={8}}\multirow{8}{*}{1/2(2T+1)} & \tagpdfsetup{table/multirow={8}}\multirow{8}{*}{1/2(2T+1)} &           \\ \cline{1-4}
    0      & 0          & 0  & 0  &                                                            &                                                            &                                                            &           \\ \cline{1-4}
    0      & 0          & 0  & 0  &                                                            &                                                            &                                                            &           \\ \cline{1-4}
    0      & 0          & 0  & 0  &                                                            &                                                            &                                                            &           \\ \cline{1-4}
    0      & 0          & 0  & 0  &                                                            &                                                            &                                                            &           \\ \cline{1-4}
    0      & 0          & 0  & 0  &                                                            &                                                            &                                                            &           \\ \cline{1-4}
    0      & 0          & 0  & 0  &                                                            &                                                            &                                                            &           \\ \cline{1-4}
    0      & 0          & 0  & 0  &                                                            &                                                            &                                                            &           \\ \hline
  \end{tabular}
\end{table}
%\tagpdfsetup{table/header-rows={1,2}} would have rows 1 and 2 be header rows
\tagpdfsetup{table/header-rows={1}}
%Use \tagpdfsetup{table/header-columns={}} for header columns instead
%Put \tagpdfsetup{table/multirow={⟨number of rows comma separated ⟩} in cells spanning multiple rows
\begin{table}[ht]
  %As a continued table, use \ContinuedFloat to keep the same #
  %And use the text "Continued" inside the caption
  \ContinuedFloat
  \caption{Continued}
  %\tagpdfsetup{table/header-rows={1,2}} would have rows 1 and 2 be header rows
  \tagpdfsetup{table/header-rows={1}}
  %Use \tagpdfsetup{table/header-columns={}} for header columns instead
  %Put \tagpdfsetup{table/multirow={⟨number of rows comma separated ⟩} in cells spanning multiple rows

  \begin{tabular}{rrrlrrrr}
    k      & q          & p+ & p- & s1        & s2         & s3          & RHS       \\ \hline
    2      & 2          & 2  & 1  & 1         & 0          & 0           & 1         \\
    -T     & 0          & 1  & 1  & 0         & 1          & 0           & 0         \\
    T      & -1         & 0  & 1  & 0         & 0          & 1           & 0         \\
    -1     & 1          & -1 & 1  &           &            &             &           \\ \hline
    2(T+1) & 2          & 0  & 1  & 1         & -2         & 0           & 1         \\ \hline
    -T     & 0          & 1  & 1  & 0         & 1          & 0           & 0         \\
    T      & -1         & 0  & 1  & 0         & 0          & 1           & 0         \\
    -(T+1) & 1          & 0  & 1  & 0         & 1          & 0           &           \\ \hline
    0      & 2+2(T+1)/T & 0  & 1  & 1         & -2         & -2(T+1)/T   & 1         \\ \hline
    0      & -1         & 1  & 1  & 0         & 1          & 1           & 0         \\
    1      & -1/T       & 0  & 1  & 0         & 0          & 1/T         & 0         \\
    0      & 1-(T+1)/T  & 0  & 1  & 0         & 1          & (T+1)/T     &           \\ \hline
    0      & 2(2T+1)/T  & 0  & 1  & 1         & -2         & -2(T+1)/T   & 1         \\ \hline
    0      & -1         & 1  & 1  & 0         & 1          & 1           & 0         \\
    1      & -1/T       & 0  & 1  & 0         & 0          & 1/T         & 0         \\
    0      & -1/T       & 0  & 1  & 0         & 1          & (T+1)/T     &           \\ \hline
    0      & 1          & 0  & 1  & T/2(2T+1) & -T/(2T+1)  & -1          & T/2(2T+1) \\ \hline
    0      & 0          & 1  & 1  & T/2(2T+1) & 1-T/(2T+1) & 0           & T/2(2T+1) \\
    1      & 0          & 0  & 1  & 1/2(2T+1) & -1/(2T+1)  & 0           & 1/2(2T+1) \\ \hline
    0      & 0          & 0  & 0  & 1/2(2T+1) & 1-1/(2T+1) & -1+(T+1)/TT &           \\ \hline
  \end{tabular}
\end{table}


% Chapter 3 from the thesis template file
%   that contains an example table and figure.
\chapter{\MakeUppercase{Methods and Procedures}}
% \MakeUppercase{} is conveninent, but if it causes undesired
%behavior (perhaps due to acronyms or math), you can just manually capitalize your chapter title
% \chaper{GENERAL INTRODUCTION}

This is the opening paragraph to my thesis which
explains in general terms the concepts and hypothesis
which will be used in my thesis.

With more general information given here than really
necessary.

\section{Introduction}

Here initial concepts and conditions are explained and
several hypothesis are mentioned in brief.

As can be seen in \autoref{nothing} it is truly
obvious what I am saying is true.

\begin{table}[h!tb] \centering

    \caption[Short caption for List of Figures/ Tables]{This table shows a standard empty table \autocite{kleeHellyTheoremIts1963}. Remove the square bracketed information to get longer captions in the LOT/ LOF }
    \label{nothing}

    \vspace{ 2 in}
\end{table}

\subsection{Hypothesis}

Here one particular hypothesis is explained in depth
and is examined in the light of current literature.

This can also be seen in \autoref{moon} that the
rest is obvious.

\begin{figure}[h!tb] \centering

    \includegraphics[alt={Alt text describing}]{Images/dc5.jpg}
    \caption[Short caption for List of Figures/ Tables]{This table shows a standard empty figure. Remove the square bracketed information to get longer captions in the LOT/ LOF}
    \label{moon}
\end{figure}

\subsubsection{Parts of the hypothesis}

Here one particular part of the hypothesis that is
currently being explained is examined and particular
elements of that part are given careful scrutiny.

% Below \subsubsection
% Sectional commands: \paragraph and \subparagraph may also be used

\subsection{Second Hypothesis}

Here one particular hypothesis is explained in depth
and is examined in the light of current literature.

\subsubsection{Parts of the second hypothesis}

Here one particular part of the hypothesis that is
currently being explained is examined and particular
elements of that part are given careful scrutiny.

%\addtocontents{toc}{\protect\newpage} % Adds \newpage in "\tableofcontents"
\section{Criteria Review}

Here certain criteria are explained thus eventually
leading to a foregone conclusion as can be seen in
\autoref{nevermore}.
\begin{table}[h!tb] \centering
    \captionsetup{width=2in}
    \caption{This table shows a standard table with a limited caption width}
    %
    %use \tagpdfsetup{table/tagging=presentation} if and only if the table
    %is for alignment/presentation purposes and is not a real table
    %
    %\tagpdfsetup{table/header-rows={1,2}} would have rows 1 and 2 be header rows
    \tagpdfsetup{table/header-rows={1}}
    %Use \tagpdfsetup{table/header-columns={}} for header columns instead
    %Put \tagpdfsetup{table/multirow={⟨number of rows⟩}} in cells spanning multiple rows
    %Note, multicolumn works automatically (no extra commands necessary)

    \begin{tabular}{c|c|c}
        Header & head & head \\
        leg    & leg  & leg  \\
        leg    & leg  & leg  \\
        leg    & leg  & leg  \\
    \end{tabular}
    \label{nevermore}
    \vspace{ 2 in}
\end{table}


\part{Let's have a part page}
% Chapter 4 from the standard thesis template
%   that contains an adv. example table and figure.
%\addtocontents{toc}{\protect\newpage}
\chapter{RESULTS}

This is the opening paragraph to my thesis which
explains in general terms the concepts and hypothesis
which will be used in my thesis.

With more general information given here than really
necessary.

\section{Introduction}

Here initial concepts and conditions are explained and
several hypothesis are mentioned in brief.

Of course, data on this as seen in \autoref{data}
is few and far between.

\begin{table}[h!tb] \centering
  \caption{Moon Data}
  \label{data}
  % Use: \begin{tabular{|lcc|} to put table in a box
  %use \tagpdfsetup{table/tagging=presentation} if and only if the table
  %is for alignment/presentation purposes and is not a real table
  %
  %\tagpdfsetup{table/header-rows={1,2}} would have rows 1 and 2 be header rows
  \tagpdfsetup{table/header-rows={1}}
  %Use \tagpdfsetup{table/header-columns={}} for header columns instead
  %Put \tagpdfsetup{table/multirow={⟨number of rows⟩}} in cells spanning multiple rows
  %Note, multicolumn works automatically (no extra commands necessary)
  \begin{tabular}{lcc} \hline
    \textbf{Element} & \textbf{Control} & \textbf{Experimental} \\ \hline
    Moon Rings       & 1.23             & 3.38                  \\
    Moon Tides       & 2.26             & 3.12                  \\
    Moon Walk        & 3.33             & 9.29                  \\ \hline
  \end{tabular}
\end{table}

\subsection{Hypothesis}

Here one particular hypothesis is explained in depth
and is examined in the light of current literature.

Or graphically as seen in \autoref{mgraph}
it is certain that my hypothesis is true.

\begin{figure}[h!tb] \centering

  \includegraphics[alt={Picture of Durham you know}]{Images/dc5}

  \caption{Durham Centre}
  \label{mgraph}
\end{figure}

\subsubsection{Parts of the hypothesis}

Here one particular part of the hypothesis that is
currently being explained is examined and particular
elements of that part are given careful scrutiny.

% Below \subsubsection
% Sectional commands: \paragraph and \subparagraph may also be used

\subsection{Second Hypothesis}

Here one particular hypothesis is explained in depth
and is examined in the light of current literature.

\subsubsection{Parts of the second hypothesis}

Here one particular part of the hypothesis that is
currently being explained is examined and particular
elements of that part are given careful scrutiny.

%similar alt text works with e.g. \begin{tikzcd} or \begin{picture}
%a tikz picture that will not move
\begin{center}
  \captionsetup{type=figure}%if you don't using floating figures, you must tell latex that it is a figure
  \begin{tikzpicture}[alt={tikz graphic that works natively on the June version}]
    \draw[gray, thick] (-1,2) -- (2,-4);
    \draw[gray, thick] (-1,-1) -- (2,2);
    \filldraw[black] (0,0) circle (2pt) node[anchor=west]{A Tikz Picture that stays put! };
  \end{tikzpicture}
  \caption{A caption for a tikz picture}
\end{center}
Some more text here you see.
%a floating tikz instead
\begin{figure}[tb]
  \begin{tikzpicture}[alt={tikz graphic that works natively on the June version that floats}]
    \draw[gray, thick] (-1,2) -- (2,-4);
    \draw[gray, thick] (-1,-1) -- (2,2);
    \filldraw[black] (0,0) circle (2pt) node[anchor=west]{A Tikz Picture that floats! };
  \end{tikzpicture}
  \caption{A caption for a tikz picture}
\end{figure}

\section{Criteria Review}

Here certain criteria are explained thus eventually
leading to a foregone conclusion.

\include{Body/chapter5}
%Pulls the appendices and bibliography out of any Parts
\bookmarksetup{startatroot}
\chapter*{BIBLIOGRAPHY}
\addToTOCWithoutChapter{BIBLIOGRAPHY}
\printbibliography[heading=none]
% Appendix1 file from standard thesis template

% Use these lines for multiple appendices

\appendixtitle
\appendix

%%%%%%%%%%%

%Use this line instead for a single appendix
%\singleappendixtitle
%%%%%%%%%%

\chapter{ADDITIONAL MATERIAL}
\finishAppendixSetup%must come immediately after the first appendix chapter command

This is now the same as any other chapter except that
all sectioning levels below the chapter level must begin
with the *-form of a sectioning command.

\section*{More stuff}
\begin{equation}
	x^2+z=yz
\end{equation}
\begin{figure}
	\caption{Something something.}
\end{figure}
\begin{table}[h!tb] \centering
	\caption[This table shows a standard non-empty table. Please check the code caption for extended instructions]{This table shows a standard empty table. In case of long captions, we want to use the long caption as the description to the table and image but not use it in the table of contents and list of figures/ tables. In order to do this, there are two captions which have been provided, remove the first square bracket options if there is only one small caption. You can use citations like this to}
	\begin{tabular}{ll}
		Bach      & Cello Suite Number 1  \\
		Beethoven & Cello Sonata Number 3 \\
		Brahms    & Cello Sonata Number 1
	\end{tabular}
	\label{nothingagain}

	\vspace{ 2 in}
\end{table}
Supplemental material.
\begin{figure}[b]
	\begin{subfigure}[c]{0.495\textwidth}
		\centering\includegraphics[alt={sample image},width=0.99\textwidth]{example-image-c}%
		\subcaption{\label{fig:2a}}
	\end{subfigure}
	\begin{subfigure}[c]{0.495\textwidth}
		\centering{\includegraphics[alt={sample image},width=0.99\textwidth]{example-image-c}}%
		\subcaption{\label{fig:2b}}%
	\end{subfigure}%
	\caption{A figure with two subfigures: (a) first subfigure; (b) second subfigure.\label{fig:2}}
\end{figure}

\begin{algorithm}
	\caption{Score Algorithm}
	\begin{algorithmic}[1]
		\State {\textbf{Input: }{$s$ is a sensor }}
		\Statex
		\For{$j\in \{1,2,\ldots,15\}$}
		\State Randomly choose $5$ days
		\For{$x\in \{1,2,\ldots,1000\}$}
		\parstate{Set $a$ to be something in this very long state that will have to be wrapped quite possibly around and around and around}
		\EndFor
		\EndFor
	\end{algorithmic}
\end{algorithm}


% % Instruction for single appendix check instruction in Appendix/appendix1.tex on top of the file
% An example second appendix from the example thesis thesis.tex.
\chapter{STATISTICAL RESULTS}

This is now the same as any other chapter except that
all sectioning levels below the chapter level must begin
with the *-form of a sectioning command.

\section*{Supplemental Statistics}

More stuff.
\begin{equation}
    \label{greatAppEq}
    z^2-a=2y
\end{equation}
You see here I reference \autoref{greatAppEq}
\begin{theorem}
    Appendices are the best part.
    \label{appThm}
\end{theorem}
Now, I reference \autoref{appThm}.
\tagtool{flush-floats=subsection}% to keep floats where they are supposed to be in the tagging tree

\end{document}

% IMPORTANT NOTES
% TABLE OF CONTENTS : First consider if the bib-no-break option above does what you want.
% TOPIC 1:  If you need a page break follow the steps below
% step1
% check before which chapter in the table of contents you want a page break
% step 2
% go the folder "body". There open the chapter tex file that you noted needed page break in the table of contents..
% step 3
% insert  \addtocontents{toc}{\protect\newpage} before the first line i.e. before the line \chapter{RESULTS}.

%%%%%%%%%%%%%%%%%%%%%%%%%%%%
% IN ORDER TO MAKE spacing changes in the title page got to the section in the isuthesis.cls file
% that starts with \long\def\maketitle{\begin{titlepage} and you can change the vskip/vspace amounts throughout

% use \caption{} for all captions of figures and tables, where the captions are not too long.

% Use \caption[]{} with the square brackets for short caption of figure or table that goes into the list of tables and list of figures, and the curly brackets can have long captions which go with the figure/ table.

